\documentclass{article}
\usepackage{proof}
\usepackage{bussproofs}
\usepackage{xcolor}
\usepackage{url}
\usepackage{framed}
\usepackage{amsmath, amsfonts, amssymb}
\usepackage{mathtools}
\usepackage{minted}

\usepackage{hyperref}
\hypersetup{
 pdfborder={0 0 0},
 colorlinks=true,
 linkcolor=blue,
 urlcolor=blue,
 citecolor=blue
}
\usepackage{cleveref}


\newcommand{\rem}[1]{\textcolor{red}{[#1]}}
\newcommand{\ed}[1]{\textcolor{blue}{#1}}
\newcommand{\ct}[1]{\rem{#1 --ct}}

\begin{document}
\title{Comprehensive Exam Report 1 - Foundations of Theorem Provers}
\author{Arjun Viswanathan}
\date{}
\maketitle

\section{Introduction}
\label{sec:intro}
	In this report, we will compare two historic 
	interactive theorem provers - LCF (Logic for 
	Computable Functions) and Automath. Both proof 
	assistants are driven by a small kernel, and 
	reduce proof checking to type checking but 
	their type systems vary significantly. 
	Automath is driven by the Curry-Howard 
	Isomorphism whereas type checking in LCF 
	is reduced to type checking in its 
	implementation language of abstract data 
	type instances. We take a look at the 
	fundamentals of these theorem provers and how 
	these have manifested in modern proof 
	assistants that have been inspired by these 
	two systems.

\section{De Bruijn's Automath}
\label{sec:automath}
	N. G. de Bruijn introduced the Automath 
	language in the 
	1968~\cite{deBruijn1983, 10.1007/BFb0060623} 
	as a means to formalize mathematics in 
	computers so that it can be mechanically
	checked. This resulted in the Automath 
	project that, among other things, resulted
	in the development of a proof-checker 
	for the Automath language~\cite{5df50ef4ddff4ff091125a511f563319, 
	c398556a074e49a193d5035ae32aaa79}.

	Inspired by colleague A. Heyting, De Bruijn 
	realized that in a typed language, logical 
	propositions can also be considered types 
	of their corresponding proofs, which would 
	constitute terms of the respective types.
	This crucial insight allowed logic to be 
	done in a typed language. This 
	correspondence between propositions 
	and types, and proofs and terms was 
	discovered independently by H. B. Curry 
	and W. Howard and is thus termed as 
	the Curry-Howard(-de Bruijn) isomorphism.
	
	\subsection{The Curry-Howard Isomorphism}
	\label{sec:curry}
	The Curry-Howard Isomorphism encapsulates 
	the correspondence between proofs and 
	terms/programs and between propositions
	and types. In doing so, it equates the 
	notion of proving, which is generally 
	considered hard, to computing.
	Initially, Haskell Curry noticed the 
	correspondence between minimal 
	implication logic and a simple 
	$\lambda-$ calculus, which was extended
	by William 
	Howard~\cite{Howard1995-HOWTFN}.
	
	Minimal implication logic is a subset
	of propositional logic with only 
	implication and its introduction 
	and elimination rules. It can be 
	characterized by the following 
	natural deduction rules:\\ \\
	\begin{math}
		\infer[(\texttt{ax})]{\Delta \vdash A}{}
		\ \ \ \ 
		\infer[(\to\texttt{-in})]{\Delta \vdash A \to B}
		{\Delta \cup \{A\} \vdash B}
		\ \ \ \ 
		\infer[(\to\texttt{-el})]{\Delta \vdash B}
		{\Delta \vdash A \to B & \Delta \vdash A}
	\end{math} 
	\\ \\
	Untyped $\lambda-$calculus gives a basic 
	mechanism for dealing with functions, making 
	it the foundation of functional programming.
	The syntax of terms in untyped $\lambda-$
	calculus is as follows:
	\begin{align*}
		\Lambda := \nu\ |\ (\lambda \nu.\Lambda)
		\ |\ (\Lambda\ \Lambda)
	\end{align*}  
	where $\nu$ represents a variable, $\lambda$ 
	is used to denote function abstraction, 
	and two juxtaposed terms represent the 
	application of the first one to the second.
	Given a context $\Gamma$ that gives types 
	to variables, the rules for deriving 
	types of $\lambda$-terms are as follows: \\ \\
	\begin{math}
		\infer[(\texttt{var})]{\Gamma \vdash x : A}
		{x : A \in \Gamma}
		\ \ \ \ 
		\infer[(\texttt{abs})]
		{\Gamma \vdash \lambda x. M : A \to B}
		{\Gamma, x : A \vdash M : B}
		\ \ \ \ 
		\infer[(\texttt{app})]{\Gamma \vdash M\ N : B}
		{\Gamma \vdash M : A \to B & \Gamma \vdash N : A}
	\end{math} 
	\\ \\
	Curry noticed an isomorphism between these 
	2 systems of rules. Howard extended this to 
	predicate logic and Heyting arithmetic. The 
	Curry-Howard isomorphism for the systems 
	of natural deduction for minimal implication 
	logic and the typing rules of untyped lambda 
	calculus terms can be stated as follows:\\

	\noindent If $\Sigma$ is a derivation in natural 
	deduction with conclusion $\Delta \vdash A$,
	then $\bar{\Delta} \vdash \bar{\Sigma} : A$,\\
	
	\noindent where $\bar{\Sigma}$ is a 
	$\lambda-$term,	and $\bar{\Delta}$ is a 
	context such that for each $B \in \Delta$, 
	there exists some $x$ such that 
	$x : B \in \bar{\Delta}$. There 
	is also a mapping from typable $\lambda-$ terms
	to derivations that completes the isomorphism.
	
	\subsection{Automath to Coq}
	\label{sec:coq}
	The Curry-Howard Isomorphism is prevalent in 
	Automath and subsequent theorem provers that 
	were motivated by Automath. One of the most 
	popular of this lineage of Automath-style 
	theorem provers is Coq. In Coq, a theorem is a 
	formula which is proven using functions 
	called tactics. To prove a conjecture in Coq, 
	it suffices to (and is necessary to) provide 
	to Coq, a term whose type is the proposition 
	being proved. In most cases, this is done 
	using tactics that continuously modify the 
	proof term, so it has no holes. Initially, 
	the entire term is a hole that has the goal's 
	type, and tactics break this down into smaller 
	and smaller holes, until all of them are 
	filled.
	
	In addition to Automath, de Bruijn's 
	contribution to the interactive theorem 
	proving world includes an answer to the 
	question `Why is a mechanical proof, 
	more dependable than a written one?'.
	A proof assistant satisfies the 
	``de Bruijn criterion''~\cite
	{10.5555/778522.778527} if it generates 
	proof terms that can be checked 
	independently from the system by a simple 
	program that a skeptical user could write 
	him/herself.
	
	By reducing the proof checker of Coq to 
	a type checker for propositions, the 
	Curry-Howard isomorphism helps Coq
	fulfill the de Bruijn criterion. Type 
	checkers are small and 
	straightforward programs that are - 
	in principal - easier to check 
	externally. 
	While this was likely 
	more true for a simpler system like 
	Automath than it is for a modern 
	one like Coq (which has a much 
	more complicated type system), it 
	remains true that type-checkers are 
	relatively simple in the space of 
	all programs, which contributes to 
	the widespread acceptance of 
	properties proved in Coq.
	
	Over the years, Coq has been enhanced 
	with complicated tactics and even 
	tactic languages. The soundness of 
	these tactics doesn't affect the 
	soundness of the Coq proof system, 
	because it is only concerned with 
	the well-typedness of the proof 
	term corresponding to a proposition,
	not with how the term was constructed. 
	A buggy tactic, by definition, would 
	construct an ill-typed proof term,
	which wouldn't pass through the 
	Coq type-checker.
	
	To demonstrate how Coq utilizes 
	the Curry-Howard Isomorphism, 
	consider the following example, 
	motivated by Chapter 9 of the 
	Logical Foundations 
	book~\cite{Pierce:SF1}.
	
	Logical conjunction in Coq is 
	defined as a proposition that 
	can be constructed only using 
	the two conjuncts. It is defined
	as an inductive type with 
	only one constructor:
	\begin{minted}{coq}
Inductive and (P Q : Prop) : Prop :=
	| conj : P -> Q -> and P Q.
	\end{minted}
	
	Noting that $\land$ is the infix 
	notation for conjunction, 
	a proof of \texttt{P $\land$ Q -> Q} 
	would involve using tactics that
	extract the evidence of \texttt{Q}
	from that of \texttt{P $\land$ Q}.
	Specifically:
	\begin{minted}{coq}
Theorem ex : forall P Q, P /\ Q -> Q.
Proof.
  intros P Q HPQ. 
  destruct HPQ as [HP HQ].
  apply HQ.
Qed.
	\end{minted}
	The proof script between 
	\texttt{Proof.} and \texttt{Qed.}
	consists of tactics that complete 
	the proof. The \texttt{intros}
	tactic localizes the variables 
	\texttt{P} and \texttt{Q} and 
	evidence \texttt{HPQ} for 
	\texttt{P $\land$ Q}. The 
	propositions-as-types correspondence
	is on display here with \texttt{ex}
	having type 
	\texttt{forall P Q, P $\land$ Q -> Q}
	and \texttt{HPQ} having type
	\texttt{P $\land$ Q}. The 
	\texttt{destruct} tactic separates 
	the proof of the conjunction into 
	proofs of its conjuncts, and the 
	\texttt{apply} tactic plugs the 
	right conjunct into the hole 
	representing the goal. Finally,
	the \texttt{Qed} command calls 
	Coq's type-checker to verify that 
	the term constructed by the 
	tactics has the type of the 
	lemma trying to be proved.
	
	To see the proof term that has 
	\texttt{forall P Q, P $\land$ Q -> Q}
	as its type, consider the following
	equivalent definition of \texttt{ex}.
	
\begin{minted}{coq}
Definition ex : forall P Q, P /\ Q -> P :=
  fun P Q HPQ => 
    match HPQ with
	  | conj P Q => Q
    end.
\end{minted}
	This definition defines a function that 
	behaves exactly like the tactics do - 
	given the evidence for \texttt{P $\land$ Q},
	it destructs this evidence by pattern
	matching on all things a conjunction can 
	be. From our definition above, we only 
	know that it can be a combination of 
	the proofs of its conjuncts. From the 
	only case of the match, it returns the 
	right proof term. This function 
	definition defines the exact proof term 
	that is constructed using the tactics 
	above, thus demonstrating the 
	proofs-as-terms correspondence.

\section{Milner's LCF}
\label{sec:lcf}
	Dana Scott developed a logic for 
	reasoning about computable functions at 
	Stanford, and Robin Milner who visited 
	Stanford and worked with Scott, 
	later built the Edinburgh LCF
	(Logic for Computable Functions) 
	system~\cite{Gordon1978EdinburghLA}.
	The LCF system provided a functional 
	programming language ML, designed to 
	implement the LCF logic and write proofs 
	in it. ML abbreviates meta language, 
	to distinguish itself from the object 
	language its encoding, that is, the 
	proof calculus. The features of ML 
	include a polymorphic type system, 
	functions as first-class objects 
	that can be passed around like terms, 
	and abstract datatypes - data 
	structures that restrict the internal 
	data to access via a fixed set of 
	functions. 
	
	For Edinburgh LCF,
	Milner had the idea of recording proof
	results, and not entire proofs, once a 
	proof had been checked. This avoided a 
	lot of redundancy and saved space.
	He implemented this using 
	an ML abstract datatype \texttt{theorem} 
	for defining theorems in LCF. The 
	predefined values of this ADT were 
	instances of axioms and its operations 
	were inference rules. Since this 
	\texttt{theorem} type was the only
	way to create new theorems in the LCF 
	system, type-checking would ensure that 
	the only way to create theorems 
	was via their axioms and inference 
	rules. Since the type-checker guaranteed
	that only valid theorems could be 
	created, once an instance of a theorem 
	had been created by the type-checker, 
	LCF didn't have to remember how it 
	was created, just that it was created 
	soundly. 
	
	Milner also introduced the concepts of 
	backward proofs, tactics and tacticals.
	While it was common for proofs to be 
	constructed forward - from the axioms 
	to the goal to be proved, a backward 
	approach breaks down the goals into 
	simpler (sub-)goals and eventually 
	to something resembling the axioms. A
	tactic is a function that helps in 
	this reduction, and tacticals can 
	combine tactics to perform tasks 
	such as repetitions and branching. 
	Tactics and tacticals add some 
	automation to proof procedures, but 
	they lie outside the trusted proof 
	kernel. A tactic is a function 
	operating on values of type 
	\texttt{theorem} and a tactical 
	is a combination of such functions. 
	The functions will only have the 
	intended effect on the value, if 
	they are sound transformations, 
	as deemed by the type-checker.
	
	For example, consider the 
	\textit{Modus Ponens} rule, also 
	represented as the 
	implication	elimination rule in 
	the previous section:
	\begin{align*}
		\infer[MP]{\Gamma \vdash q}
		{\Gamma \vdash p \to q & \Gamma \vdash p}
	\end{align*}
	This is a rule with two premises and a 
	conclusion. In the LCF system, this is an
	ML function with type
	\texttt{theorem} $\to$ \texttt{theorem}
	$\to$ \texttt{theorem}. It takes 
	two objects of type \texttt{theorem}
	and returns a \texttt{theorem} object.
	
	\subsection{The HOL Family}
	LCF moved to Cambridge from Edinburgh
	through Milner's collaborators. The 
	HOL (Higher-Order Logic) theorem 
	prover was created for hardware 
	verification, but its support for 
	higher-order logic made it and 
	its descendants relevant for 
	various applications. The first 
	stable version of HOL - HOL88 - was
	written in Common Lisp. HOL90 was 
	then implemented in Standard ML. 
	HOL Light was another variant 
	implemented in Caml Light that 
	prioritizes constructive logic 
	over classical logic. 
	
	As new logics
	were designed, so were proof 
	assistants that could reason about 
	them. Isabelle was born as an 
	LCF-style proof assistant for 
	Martin-L\"{o}f's constructive type 
	theory and grew into a general 
	framework to deal with multiple 
	logics. Lawrence C. Paulson 
	developed it as an LCF-like
	system that provided a meta-logic, 
	within which other logical frameworks 
	could be declared - in the previous 
	HOL systems, the logic had to be 
	implemented in ML. Higher-order
	unification along with backtracking
	is central to proof search in Isabelle. 
	While higher-order unification 
	is able to unify more expressive 
	terms such as variables 
	representing functions, it adds 
	undecidability and non-determinism 
	to the solution space. For example, 
	$f\ 3$ and $3+3$ can be unified 
	by substituting 4 possible function 
	definitions for $f$: 
	$\lambda x.x+3$, $\lambda x.3+x$, 
	$\lambda x.x+x$, $\lambda x.3+3$.
	These issues are usually avoided 
	because in practice, Isabelle 
	has to deal with first-order 
	unification more often. Additionally, 
	a subset of higher-order terms, 
	called higher-order patterns were 
	found to behave like first-order 
	terms for the purposes of unification.
	When there are a non-deterministic 
	set of unifiers, backtracking allows
	different solutions to be explored.
	
	In Isabelle, the \texttt{theorem}
	type can essentially be extended to 
	reason about a particular object logic. 
	Instances of Isabelle include 
	Isabelle/CTT (constructive type theory),
	Isabelle/IFOL (intuitionistic first-order 
	logic), Isabelle/FOL (classical FOL), 
	Isabelle/FOL with Zermelo-Fraenkel 
	set theory. The most popular of these
	is Isabelle/HOL - the Isabelle
	instance for higher-order logic.
	Isabelle/HOL became a replacement for 
	HOL due to its efficient automation 
	via a simplifier and support for 
	external first-order theorem proving 
	tools~\cite
	{10.1007/978-3-540-25984-8_28}, and 
	its vast proof libraries which include 
	a vast standard library and the 
	Archive of Formal Proofs~\cite{afp}. 
	
	Thus, the reduction of proof-checking
	to type-checking of abstract datatypes
	in the meta language is at the core 
	of LCF-style proof systems which 
	includes the entire HOL family, 
	including Isabelle which is a 
	generic logical framework that 
	can be instantiated to do proof
	checking in particular logics.
	

\section{A Comparison - Automath vs LCF}
  	While there isn't evidence that the 
  	Automath project was influenced by 
  	the LCF project (to the contrary, 
  	their dates overlap), the 
  	\textit{LCF approach}, credited 
  	to Edinburgh LCF is the approach
  	followed by most - if not all - 
  	proof assistants today. 
  	Theorem provers that follow the 
  	LCF approach implement a small 
  	proof kernel that facilitates 
  	the creation of theorems. This 
  	proof kernel usually ensures 
  	soundness by some form of 
  	type-checking. In this sense, 
  	Automath, and its successors also 
  	follow the LCF approach with 
  	type-checking proof terms to have 
  	types as propositions, using the 
  	Curry-Howard Isomorphism. Some works 
  	claim that Coq is a descendant of 
  	LCF~\cite
  	{DBLP:journals/corr/abs-1907-02836}. 
  	On the other hand, the de Bruijn 
  	criterion - that necessitates 
  	that either proof terms generated 
  	by a proof assistant or the proof 
  	checker itself are externally 
  	checkable by a simple program - 
  	is arguably satisfied by LCF-style
  	theorem provers as well. Thus, 
  	while we look at some of the 
  	differences between Automath and 
  	LCF, and their respective successors, 
  	we will keep in mind that both 
  	introduced some principles that 
  	have become pervasive in modern 
  	proof assistants.
  	
  	While it is true that both 
  	Automath-style and LCF-style
  	systems reduce proof-checking
  	to type checking, the type-checkers 
  	and indeed the type systems, 
  	of both tools are quite different.
  	In LCF and HOL systems, the 
  	type-checker of the implementation 
  	language or the meta-language 
  	guarantees the soundness of the
  	proof assistant. For Automath-style
  	ITPs, the Prop type and the 
  	type-checking algorithm are 
  	implemented within the 
  	implementation language. For 
  	instance, the implementation 
  	code of Coq describes what it 
  	means to be a Prop and also 
  	how a type-checker would check 
  	terms of type Prop. This distinction
  	offers some trade-offs. On the one 
  	hand, in LCF, the trust-base is 
  	the type-checker of the entire
  	ML language, whereas in Automath, 
  	the trust-base is a type-checker
  	that is implemented inside the 
  	meta language. In principle, this 
  	could be smaller than the type
  	checker of the entire language.
  	In practice, while this is a 
  	clear difference in approach,
  	I have not seen this argument 
  	being made to claim that one tool 
  	is better than the other. 
  	Additionally, LCF has a single 
  	\texttt{theorem} type and proof
  	objects i.e. instances of type
  	\texttt{theorem} cannot be 
  	inspected or manipulated. Milner
  	intended for it to be this way
  	to make LCF space efficient - 
  	once a proof was type-checked, 
  	it only mattered \textit{what}
  	was proven, not \textit{how} it 
  	was proven. In Automath, on the 
  	other hand, these proof objects 
  	can be manipulated and pattern 
  	matched on by advanced users.
  	For instance, in Coq, proof terms 
  	can be pattern matched on by 
  	advanced users who are able to 
  	manipulate Coq's OCaml code.
  	
  	Automath and its descendents are 
  	based on Martin-L\"{o}f's dependent 
  	type theory, in contrast to the
  	simple type theory implemented by  
  	LCF and most of	its descendants. 
  	While polymorphic 
  	types allow types to be 
  	parameterized by other types 
  	types - such as \texttt{list}
  	$ \alpha$ where $\alpha$ can be 
  	any type - dependent types allow 
  	types to \textit{depend} on 
  	types or values. For instance, 
  	consider the 
  	$\texttt{cons}_{\mathbb{N}}$
  	function for \texttt{list} 
  	$\mathbb{N}$ (lists of natural 
  	numbers) that inserts elements into 
  	lists. \texttt{cons} has type 
  	$\mathbb{N}	\to \texttt{list } 
  	\mathbb{N}$. For example, 
  	inserting the element $1$ into 
  	the empty list gives the 
  	singleton list with $1$, expressed
  	as $\texttt{cons}_{\mathbb{N}}\ 
  	1\ [\ ]\ = [1]$. We would need 
  	a $\texttt{cons}_{\alpha}$ for 
  	each type $\alpha$. In a 
  	dependent type system, we can 
  	type a generic \texttt{cons} 
  	function as $\Pi\ \alpha\ :\ 
  	\texttt{Type},\ \alpha \to 
  	\texttt{list}\ \alpha \to 
  	\texttt{list}\ \alpha$. The 
  	$\Pi$ indicates that the 
  	second and third arguments 
  	to \texttt{cons} \textit{depend}
  	on the first, which is a type, 
  	or a value of type \texttt{Type} 
  	(types are often considered 
  	values in a hierarchy
  	of types in dependently typed 
  	systems). Dependently-typed 
  	systems allow for more expressive
  	types that could potentially 
  	avoid code reuse. Additionally, 
  	the type system could catch 
  	certain bugs at compile time
  	that might not be as 
  	straightforward to do with 
  	simply-typed checkers. For 
  	instance, consider the 
  	vector type $\texttt{vec}\ 
  	:\ \texttt{Type} \to \mathbb{N}
  	\to \texttt{Type}$. Here, the 
  	first argument specifies the 
  	type of elements in the vector, 
  	while the second is a natural 
  	number representing the 
  	length of the vector. Not 
  	only can we define a generic
  	\texttt{append} function for 
  	such vectors, we can also 
  	enforce a constraint on the 
  	length of the resultant vector
  	in the \texttt{append} function:
  	$\texttt{append}\ :\ \Pi\ 
  	(\alpha\ :\ \texttt{Type})\ 
  	(m\ n\ :\ \mathbb{N}),\ 
  	\texttt{vec}\ \alpha\ m \to 
  	\texttt{vec}\ \alpha\ n \to 
  	\texttt{vec}\ \alpha\ (m + n)$.
  	Given that the length of the 
  	first vector is $m$ and that 
  	that of its	second is $n$, the 
  	type system	can enforce at compile 
  	time that the length of the result 
  	of appending the vectors is 
  	$m + n$. Such a constraint 
  	must be expressed externally 
  	in a simply-typed system. For 
  	example, John 
  	Harrison~\cite{10.1007/11541868_8} 
  	modeled vectors $\texttt{vec}\ 
  	\alpha\ n$ in HOL as functions 
  	of type	$N \to \alpha$ where $N$ 
  	is a type with exactly $n$ values. 
  	The examples for dependent 
  	types were taken from 
  	Chapter 2 of Theorem Proving 
  	in Lean~\cite{avigad2016theorem}.
  
	LCF implements a classical logic
	in comparison to Automath's 
	constructive logic. Every 
	constructive proof is classically 
	valid. Classical logic can 
	be considered as constructive 
	logic with the added 
	\textit{law of the excluded 
	middle} that states that for 
	every proposition $A$, 
	$A \lor \neg A$ holds. This can 
	also be stated as the 
	double negation elimination 
	law or as proving a 
	proposition by contradiction. As
	such, some propositions might be 
	easier to prove in classical 
	logic. A common example is a 
	classical proof of the 
	following lemma. 
	
	\textbf{Lemma : } \textit{There are 
	irrational numbers $a$ and $b$ 
	such that $a^b$ is rational.}\\
	\textbf{Proof. }
	$\sqrt{2}^{\sqrt{2}}$ is either 
	rational or irrational, by the 
	law of the excluded middle. \\
	If it's rational, then 
	$a = b = \sqrt{2}$. \\
	If, $\sqrt{2}^{\sqrt{2}}$ is 
	irrational, we know 
	$(\sqrt{2}^{\sqrt{2}})^{\sqrt{2}} = 
	2$. Then, $a = \sqrt{2}^{\sqrt{2}}$
	and $b = \sqrt{2}$.
	
	Constructively, this proof is 
	more complicated and involves 
	proving that $\sqrt{2}^{\sqrt{2}}$
	is irrational. However, 
	constructive logic has a direct 
	relation to computation, which 
	is, for instance, demonstarted 
	by the Curry-Howard Isomorphism.
	
	It is possible to do classical 
	reasoning in some Automath-based
	systems, and constructive reasoning 
	in LCF-based ones. For instance, 
	Isabelle was first created as 
	a tool to reason in constructive 
	logic, and its first object-logic
	was Isabelle/CTT,based on 
	constructive type theory. While
	Coq's logic is fully constructive,
	classical axioms such as the 
	law of the excluded middle and the 
	axiom of choice can be added as 
	additional axioms to Coq to do 
	classical reasoning. Since 
	constructive proofs are 
	programs, extraction of 
	programs that are correct
	by constructions from proof 
	assistants like Coq is relatively 
	straightforward.
	
	While these theoretical aspects 
	are quite fundamental in 
	differentiating LCF and Automath 
	and their respective descendants, 
	more practical aspects might 
	play a role in choosing between 
	modern-day proof assistants. 
	Isabelle, one of the most 
	popular LCF-style provers today,
	offers a structured proof language
	called Isar, and Isabelle/HOL, 
	its most popular variant,
	offers on top of its large 
	standard library, the Archive
	of Formal Proofs~\cite{afp} - a 
	comprehensive collection of 
	proofs contributed by the 
	Isabelle community. In fact, 
	these vibrant communities for 
	each proof assistant have 
	built extensive libraries 
	for each of them over the years 
	that might make one more 
	attractive than the other. For 
	example, the Coq proof assistant
	has plenty of libraries and 
	verification projects available 
	for reuse~\cite{coqlib}, and 
	the Lean Theorem Prover also has 
	an active user community that 
	maintains a library for 
	mathematics caleld 
	mathlib~\cite{10.1145/3372885.3373824}.
	Other considerations may 
	include utilities such as 
	independent tactic languages, 
	for example, Coq's 
	LTAC~\cite{10.1007/3-540-44404-1_7}.
	
  	
\bibliographystyle{abbrv}
\bibliography{bib1}

\end{document}