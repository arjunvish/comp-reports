\documentclass{article}
\usepackage{proof}
\usepackage{bussproofs}
\usepackage{xcolor}
\usepackage{url}
\usepackage{framed}
\usepackage{amsmath, amsfonts, amssymb}
\usepackage{mathtools}
\usepackage{minted}

\usepackage{hyperref}
\hypersetup{
 pdfborder={0 0 0},
 colorlinks=true,
 linkcolor=blue,
 urlcolor=blue,
 citecolor=blue
}
\usepackage{cleveref}


\newcommand{\rem}[1]{\textcolor{red}{[#1]}}
\newcommand{\ed}[1]{\textcolor{blue}{#1}}
\newcommand{\ct}[1]{\rem{#1 --ct}}

\begin{document}
\title{Comprehensive Exam Report 2 - Relating Higher Order and First-Order Logics}
\author{Arjun Viswanathan}
\date{}
\maketitle

\section{Introduction}
\label{sec:intro}
	In this report, we will discuss the internals 
	of two tools that use ATPs to provide 
	automation to ITPs - SMTCoq and Sledgehammer's
	SMT integration. Specifically, we will look
	at the following aspects of each tool. 
	For SMTCoq, we will study 
	how proof by reflection is used to 
	\textit{reflect} proofs from an SMT solver
	to Coq's logic, and the data structures 
	used to facilitate this reflection.
	Sledgehammer provides a translation of 
	a fragment higher-order logic into 
	first-order logic, so that 
	SMT-solvers can reason about them. We will
	look into this translation in detail.

\bibliographystyle{abbrv}
\bibliography{bib2}

\end{document}