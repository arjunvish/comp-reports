\documentclass{article}
\usepackage{proof}
\usepackage{bussproofs}
\usepackage{xcolor}
\usepackage{url}
\usepackage{framed}
\usepackage{amsmath, amsfonts, amssymb}
\usepackage{mathtools}
\usepackage{minted}
\usepackage{stmaryrd}

\usepackage{hyperref}
\hypersetup{
 pdfborder={0 0 0},
 colorlinks=true,
 linkcolor=blue,
 urlcolor=blue,
 citecolor=blue
}
\usepackage{cleveref}


\newcommand{\rem}[1]{\textcolor{red}{[#1]}}
\newcommand{\ed}[1]{\textcolor{blue}{#1}}
\newcommand{\ct}[1]{\rem{#1 --ct}}

\begin{document}
\title{Comprehensive Exam Report 2 - Relating Higher Order and First-Order Logics}
\author{Arjun Viswanathan}
\date{}
\maketitle

\section{Introduction}
\label{sec:intro}
	In this report, we will discuss the 
	internals of two tools that use 
	automatic theorem provers (ATPs) to 
	provide automation to interactive 
	theorem provers (ITPs) - 
	SMTCoq~\cite{DBLP:phd/hal/Keller13} 
	and Sledgehammer's SMT solver 
	integration~\cite{bohme}. Specifically, 
	we will look at the following aspects 
	of each tool. For SMTCoq, we will study 
	how proof by reflection is used to 
	\textit{reflect} proofs from an SMT 
	solver to Coq's logic, and the data 
	structures used to facilitate this 
	reflection.	Sledgehammer provides a 
	translation of a fragment of 
	higher-order logic into first-order 
	logic, so that SMT-solvers can reason 
	about them. We will look into this 
	translation in detail.

\section{SMTCoq}
\label{sec:smtcoq}
	SMTCoq is a skeptical cooperation 
	between the Coq proof assistant, and 
	SAT and SMT solvers, implemented as a 
	Coq plugin. We will maintain a focus 
	on the SMT solver integration of 
	SMTCoq, noting that most features are 
	shared with the SAT	solver integration.
	
	ATPs like SMT solvers are susceptible 
	to bugs due to the large code-bases 
	used to support	their automaticity. 
	ITPs like Coq have a small trustable 
	proof kernel which would be 
	compromised if they were to trust
	external results. To avoid extending 
	Coq's trust-base, SMTCoq requires the 
	ATPs to be proof-producing, and uses 
	Coq's computational capabilities 
	to lift these up to Coq proofs, in a 
	process called computational 
	reflection. 
	
	\subsection{Proof by Reflection}
	\label{reflect}
	\texttt{Prop} is the type of 
	constructive propositions in Coq. 
	Additionally, Coq also has a 
	Boolean type, \texttt{bool}, which 
	is computational --- it is an 
	inductive type with constructors 
	\texttt{true} and \texttt{false}. 
	As a consequence, one can 
	perform case analysis on 
	\texttt{bool} values since they 
	are decidable. Propositions 
	aren't decidable - one cannot 
	observe whether any given 
	proposition is true or false. 
	Boolean formulas are embedded 
	into \texttt{Prop}s since all 
	theorems in Coq are stated as 
	propositions. For instance, the 
	formula 
	$\top \lor a$ for some 
	proposition $a$ can be stated 
	using Coq's \texttt{Prop} type as:
\begin{minted}{Coq}
Lemma TOrA : forall (a : Prop), True \/ a.
\end{minted}
	Using the \texttt{bool} type, it 
	is stated as:
\begin{minted}{Coq}
Lemma TOrABool : forall (a : bool), orb true a = true.
\end{minted}
 	All quantified formulas have type
 	\texttt{Prop}, so the type of both
 	\texttt{TOrA} and \texttt{TOrABool}
 	is \texttt{Prop}. 
 	\texttt{True $\backslash/$ a}
 	is also a Coq proposition, since 
 	the propositional disjunction
 	(\texttt{$\backslash/$}), returns a 
 	\texttt{Prop}. Moreover, the 
 	formula \texttt{orb true a = true}, 
 	where \texttt{orb} is the 
 	Boolean disjunction in Coq, 
 	is also a proposition since 
 	equality is a predicate 
 	(a function with return 
 	type \texttt{Prop}), but in 
 	\texttt{TOrABool}, $\top \lor a$ 
 	is expressed using Coq's 
 	\texttt{bool} type, and the 
 	Boolean $a$ can be pattern-matched 
 	on. SMTCoq uses computational 
 	reflection to prove Boolean 
 	formulas (like \texttt{TOrABool})
 	in Coq using an external 
	proof-producing SMT solvers. 
	Orthogonal to reflection, 
	SMTCoq has a \texttt{Prop} to 
	\texttt{bool} conversion 
	mechanism, that is able to take 
	certain \texttt{Prop}
	goals (that look like 
	\texttt{TOrA}), and convert 
	them to \texttt{bool} formulas, 
	which can then be proved by 
	an SMT solver. The focus in 
	this report, however,  
	is on SMTCoq's reflection 
	mechanism.
	 
	Coq's Calculus of Inductive 
	Constructions (CIC) is a 
	$\lambda-$calculus that has a 
	reduction mechanism for terms. The
	reduction rules form a strongly 
	normalizing system. The calculus's
	\textit{conversion rule} is an 
	important rule that allows different 
	terms to have the same type as 
	long as the types have the same 
	normal forms. For instance, for some 
	predicate $P$ over natural numbers, 
	a proof of $P(10)$ is also a proof 
	of $P(5*2)$, and of $P(20-10)$. Due 
	to the conversion rule, computations 
	can be used in Coq's reasoning and 
	proof terms can be found simply by 
	computing normal forms of types.
	
	A proposition in Coq is proved by 
	providing to Coq's type checker, a 
	proof term that inhabits the 
	proposition, or has the proposition 
	as its type. Automaticity can be 
	added to Coq by implementing 
	methods in Coq, that find the right 
	proof term for propositions --- a 
	process called \textit{proof search}. 
	Such a method could use an external 
	solver, such as an ATP, to guide 
	proof search. However, ATPs are 
	typically used to provide automatic 
	proofs without justification - 
	answering the question `is F 
	provable?' with	yes/no. To use the 
	result of an SMT solver without 
	compromising Coq's kernel the ATP 
	would have to produce, in addition 
	to a result, a \textit{proof trace} 
	that justifies the result. This 
	trace will guide the proof search 
	in Coq in a process called 
	\textit{proof reconstruction}. 
	Reconstruction can be implemented 
	as a meta-procedure. For example, 
	reconstruction of SMT solver proofs 
	in Isabelle/HOL are done external to 
	the prover's logic~\cite{bohme}. 
	This reconstruction can also be 
	expressed within the prover's logic 
	by a process called 
	\textit{computational reflection}, 
	which relies on the previously 
	mentioned reduction mechanism and 
	the conversion rule.

	SMT solvers prove formulas over 
	sorted (typed) terms and these 
	sorts correspond to types in 
	Coq that are generally 
	specified by the Coq standard 
	library. Terms of these library 
	types represent a \textit{shallow 
	embedding} of SMT terms within 
	Coq and SMTCoq is used to prove 
	Coq propositions over such 
	shallow terms. Since SMT solver 
	proofs are pretty large, a much 
	more efficient embedding is 
	implemented as data structures 
	using machine integers and arrays 
	in Coq. This is the 
	\textit{deep embedding}. Deep 
	terms are harder to specify, 
	since features such as binders 
	and variable substitution must be 
	handled by writing functions 
	for these within Coq. These 
	features are available as part 
	of Coq's standard library 
	for shallow terms. 
	
	A proof produced by an 
	ATP is checked against the deep 
	embedding and is
	\textit{reflected} to a proof 
	in the shallow embedding. This 
	process requires an 
	\textit{interpretation function}
	which defines the meaning of 
	data structures in terms of 
	the library types. Additionally, 
	the \textit{reflection principle} 
	is a powerful lemma that relates 
	all formulas in the deep embedding
	to those in the shallow embedding. 
	The proof involves an induction 
	on the structure of the 
	data structures in order to prove 
	their similarity to 
	the shallow terms, while using 
	the interpretation function. This 
	proof is factored once and can be 
	reused by instantiating it with 
	specific formulas proved by the 
	external solvers. Work is then 
	done by Coq's computational 
	mechanism to check that the proof 
	trace from the SMT solver proves 
	the (deep) formula it claims to 
	prove. Deep and shallow 
	embeddings were first introduced 
	while discussing embedding 
	hardware description languages 
	in the HOL theorem 
	prover~\cite{10.5555/645902.672777}.
	
	Formally, reflection of proofs 
	in SMTCoq is defined as follows. 
	In Coq, we have 
	\begin{itemize}
		\item \texttt{form} --- a data 
		structure representing the deep 
		embedding of SMT formulas
		\item \texttt{bool} --- the 
		Boolean type in Coq 
		representing the shallow 
		embedding of SMT formulas
		\item $\mathbb{I}$ --- the 
		interpretation function, which 
		is a Boolean predicate over 
		\texttt{form}s (of type 
		\texttt{form} $\to$ 
		\texttt{bool})
		\item \texttt{T} --- the type of 
		proof traces
	\end{itemize}
	Reflection requires the following:
	\begin{itemize}
		\item a function 
		\texttt{check : form $\to$ T $\to$ 
		bool} implemented in Coq such that 
		\texttt{check}$\ f\ t$ is 
		\texttt{true} if trace $t$
		justifies $f$,
		where $f$ has type 
		\texttt{form}
		\item a lemma that proves the 
		correctness of \texttt{check}:
		\begin{align*}
			\texttt{check\_correct :} 
			\forall\ (f : \texttt{form})\ 
			(t : \texttt{trace}),\ 
			\texttt{check}\ f\ t\ = 
			\texttt{true} \to \\
			\mathbb{I}(f) = 
			\texttt{true}
		\end{align*}
		This is called the 
		\textit{reflection principle}.
	\end{itemize}
	The reflection principle relates 
	the computational behavior of 
	\texttt{check} to its meaning
	in the shallow embedding. For some 
	$f : \texttt{form}$ and trace $t$, 
	the proof of $\mathbb{I}(f) = 
	\texttt{true}$ is:
	\begin{center}
		$\texttt{check\_correct}\ \ 
		f\ \ t\ \ (\texttt{refl\_equal}\ \ 
		(\texttt{check}\ f\ t)\ \ 
		\texttt{true})$
	\end{center}
	where \texttt{refl\_equal} is a tactic
	that forces the Coq type checker to 
	perform an equality check between 
	$\texttt{check}\ f\ t$ and \texttt{true}
	by reduction. The reflection principle 
	is proved once and is reused for 
	each formula. However, the trade-off is 
	that the reduction mechanism and the 
	conversion rule are invoked for each 
	application of the proof, adding a 
	computational overhead. 
	
	\subsection{Deep and Shallow Embeddings}
	In the following, we present 
	a simplified version of SMTCoq's 
	deep embedding more concretely. 
	The actual embedding uses Coq's 
	machine integers to implement 
	sharing of terms and optimize 
	space, but an unoptimized version 
	is presented below for brevity. 
	
	\texttt{form}, the type of 
	formulas, looks like this:
	\begin{minted}{coq}
Inductive form : Type :=
	| Fatom (_ : atom)
	| Ftrue
	| Ffalse
	| Fand (_ : array form)
	| For (_ : array form)
	| Fimp (_ : array form)
	| Fxor (_ _ : form)
	| Fiff (_ _ : form)
	| Fite (_ _ _ : form)
	| Fnot2 (_ : form)
	\end{minted}
	where a formula can be an atom, 
	\texttt{true}, \texttt{false};
	a conjunction, disjunction, or 
	implication of two or more 
	formulas; an exclusive-or or
	equivalence of two formulas; 
	an if-then-else of three formulas, 
	or a double negation of a formula.
	
	Formulas or their negations 
	represent literals:
	\begin{minted}{coq}
Inductive lit : Type :=
	| Pos (_ : form)
	| Neg (_ : form)
	\end{minted}
	A clause is a list of literals,
	and a state is a set of clauses.
	The \texttt{form} type references 
	\texttt{atom}, the type of atoms, 
	which represents variables, 
	constants, and applications of 
	functions of different theories.
	
	\noindent The \texttt{check} 
	function is implemented as the 
	\textit{main checker}: 
	\begin{center}
		\texttt{main\_checker ($input$ : 
			state) ($trace$ : T) : bool}	
	\end{center}
	which is a function that checks 
	whether the deep embedding $input$
	corresponds to the proof 
	$trace$ from the SMT solver, 
	and returns \texttt{true} if it 
	does. The main checker splits up 
	the checking among a number of 
	\textit{small checkers}. The 
	trace is an array of \textit{steps}, 
	and each step can be independently 
	checked by a small checker. Currently, 
	there are small checkers for steps 
	that perform clause resolution, 
	conversion of formulas to conjunction 
	normal form (CNF), SMT solver 
	simplifications, and one for each 
	theory:	equality over uninterpreted 
	functions (EUF), linear integer 
	arithmetic (LIA), bit-vectors (BV), 
	and functional arrays with 
	extensionality (AX). The small
	checkers add inferred clauses to 
	the state via their respective 
	steps, and the main checker 
	checks that eventually, the empty 
	clause is added to the state, 
	signifying unsatisfiability. The 
	reflection principle is a proof of 
	correctness of the main checker:
	\begin{align*}
		\texttt{main\_checker\_correct
		: } &\texttt{$\forall$ $input$ 
		$trace$, main\_checker $input$ 
		$trace$ = true}\\
		&\texttt{$\to$ $\forall$ 
		$\mathbb{I}$,
		valid $\mathbb{I}$
		$\neg input$ = true}
	\end{align*}
	where $\mathbb{I}$ is the 
	interpretation function that 
	relates the deep formula to its 
	corresponding shallow embedding,
	and \texttt{valid} is a function 
	that specifies the correctness
	of formulas in the shallow 
	embedding. The state is 
	negated in \texttt{valid} because 
	of the duality between 
	satisfiability and validity. A
	quantifier-free formula is valid, 
	if and only if its negation 
	is unsatisfiable. 
	
	The checker is \textit{sound} : 
	when it returns \texttt{true} for 
	a particular state, the formulas 
	in the state are unsatisfiable, but 
	not \textit{complete}: when it 
	returns \texttt{false} we can't be 
	certain that the formulas in the 
	state are satisfiable. This is 
	because the SMT solver isn't 
	decidable --- given a set of 
	formulas, it could either 
	return \texttt{sat} (satisfiable), 
	\texttt{unsat} (unsatisfiable), or 
	\texttt{unknown} (`don't know').
	The proof of correctness of the main 
	checker is also composed of the 
	proofs of correctness of the small 
	checkers, and these proofs involve 
	inductions over the structures of 
	the different types presented 
	above. 

	
\section{Sledgehammer}
\label{sec:hammer}
	Isabelle~\cite{DBLP:journals/corr/cs-LO-9301106} 
	is an LCF-style system that 
	provides a meta-logic which can be 
	instantiated with other logics.
	Isabelle/HOL~\cite{10.5555/1791547}, 
	one of the most popular Isabelle 
	instantiations, implements a 
	classical higher-order logic. 
	
	Sledgehammer is
	an Isabelle/HOL component that 
	uses external ATPs to enhance 
	Isabelle/HOL with proof 
	automation. Initially, these 
	ATPs only included resolution 
	provers~\cite{10.1007/978-3-642-39799-8_1}.
	The work by Bohme et 
	al.~\cite{bohme} involved 
	extending Sledgehammer to 
	incorporate SMT
	solvers~\cite{Barrett2018} and this 
	work will be our focus for the 
	rest of this section. As with 
	SMTCoq, the SMT solvers integrated
	with Sledgehammer produce a 
	\textit{proof trace} which is 
	then \textit{reconstructed} within
	Isabelle/HOL by Sledgehammer, 
	using Sledgehammer's own internal 
	ATP --- Metis~\cite{hurd2003d}. The 
	proof trace essentially guides 
	the inference steps of the proof 
	within Isabelle/HOL.
	
	Given a conjecture $\Phi$ in 
	Isabelle/HOL, Sledgehammer 
	selects a set of facts 
	$\Gamma$ that might be relevant 
	to proving $\Phi$ and sends 
	$\Gamma \land \neg \Phi$ to the 
	SMT solver to check for 
	unsatisfiability. If the SMT 
	solver is able to conclude 
	unsatisfiability, it returns 
	a proof trace which is 
	used to guide the proof search 
	of Metis inside Isabelle/HOL.
	Since Isabelle/HOL implements 
	a higher-order logic (HOL), which 
	is more expressive than 
	the SMT solver's multi-sorted
	first order logic (MSFOL),
	the entirety of the conjecture
	$\Gamma \land \neg \Phi$ may not 
	be understandable to the SMT 
	solver. Although, parts of 
	this conjecture may be fully 
	first-order because FOL is 
	a subset of HOL. For the 
	rest of it, Sledgehammer
	has a translation from 
	higher-order to first-order
	logic which is described 
	in the rest of this section.
	
	\subsection{Higher-Order Logic}
	\label{sec:hol}
	In this section, we specify the 
	syntax of higher-order logic 
	and delegate the explanation of 
	semantics to a 
	reference~\cite{10.5555/155278}. 
	HOL consists of 
	types $\tau$ and terms $t$. 
	
	\begin{align*}
	\tau &:= \alpha\ |\ \kappa^n\ 
	\tau_1 ... \tau_n\\
	t &:= x^{\tau}\ |\ c^{\tau}\ |\ t_1\ t_2\ |\ 
	\lambda x^{\tau}.t
	\end{align*}	
	Types $\tau$ are either type
	variables $\alpha$ or 
	applications of type 
	constructors $\kappa^n$ to 
	$n$ types ($n$ is usually omitted). 
	Particular types of interest are 
	the function type constructor 
	$\to^{2}$, the Boolean type 
	--- $\texttt{bool}$ (or 
	$\texttt{bool}^0$), the type of 
	natural numbers --- \texttt{nat},
	and integers --- \texttt{int}.
	Terms are typed variables, 
	typed constants, applications 
	of terms to terms, or a typed
	$\lambda-$ abstraction. We have
	the usual Boolean constants 
	representing logical connectives
	and quantifiers. For instance, 
	logical negation, 
	$\neg^{\texttt{bool} \to 
	\texttt{bool}}$, universal 
	quantification,
	$\forall^{\alpha \to 
	\texttt{bool} \to \texttt{bool}}$, 
	and polymorphic equality,
	$=^{\alpha \to \alpha 
	\to \texttt{bool}}$ Type 
	annotations for terms are also 
	often omitted when understood
	from context.

	\subsection{Many-Sorted First-Order Logic}
	\label{sec:msfol}
	Many-sorted first-order logic extends
	first-order logic (FOL) with 
	types or sorts. We present the 
	syntax in this section and the 
	semantics are presented in
	\cite{Barrett2018}. Syntactically, 
	the components of MSFOL are sorts 
	$\sigma$, terms $t$, and 
	formulas $\phi$. Sorts are 
	atomic entities that 
	represent types. Function types 
	--- ($\sigma_1$, ..., $\sigma_n$) 
	$\to$ $\sigma$ ---
	and relation types 
	--- ($\sigma_1$, ..., $\sigma_m$)
	are defined over sorts, and 
	are types of functions and 
	predicates below. Terms and 
	formulas are specified as:
	\begin{align*}
		t &:= x^{\sigma}\ |\ 
		f^{(\sigma_1, ..., \sigma_n) \to 
		\sigma}	(t_1, ..., t_n)\\
		\phi &:= \bot\ |\ \neg \phi\ |\ 
		\phi_1 \land \phi_2\ |\ \forall 
		x^{\sigma} . \phi\ |\ | t_1 = t_2
		\ |\ P^{\sigma_1,...,\sigma_m}
		(t_1, ..., t_n)
	\end{align*}
	Terms are either sorted variables, 
	or functions applied to terms.
	Formulas are constants or logical 
	connectives applied to other 
	formulas, quantified formulas, 
	equality over terms, or predicates 
	over terms. Connectives $\lor$, 
	$\to$, $\iff$, and the existential
	quantifier $\exists$ can be 
	specified using the connectives 
	and quantifiers mentioned above.
	
	\subsection{Translation}
	\label{sec:trans}
	Sledgehammer's SMT solver 
	integration (Bohme et al.) performs 
	a translation 
	of HOL formulas to MSFOL formulas.
	This translation is sound ---
	if the translated problem is 
	valid, then the original problem 
	is also valid --- but not complete
	--- a valid problem is not 
	necessarily translated to a 
	valid one. This means that the 
	translation might not be 
	successful in translating 
	certain HOL problems which is
	expected, since HOL is a 
	much more expressive logic than 
	MSFOL. $\llbracket\ \rrbracket$
	is the translation function 
	that maps both HOL types and 
	terms to MSFOL sorts and terms,
	respectively.
	
	Since MSFOL is a subset of 
	HOL, we can see HOL as a 
	composition of 
	MSFOL-equivalent logic 
	and the rest:
	\begin{center}
		HOL = MSFOL-equiv + non-MSFOL 
	\end{center}
	The MSFOL-equiv part of HOL has 
	a straightforward transformation 
	to MSFOL:
	\begin{itemize}
	\item Only nullary type 
		constructors from HOL have
		equivalent MSFOL sorts:
		\begin{center}
			$\llbracket \kappa^{0} 
			\rrbracket = \sigma $
		\end{center}
	\item Applications of 
		Boolean connectives,
		quantifiers, equalities and 
		other predicates (functions 
		returning \texttt{Bool}) to 
		terms have corresponding 
		MSFOL formulas:
		\begin{align*}
			\llbracket False 
			\rrbracket &\cong \bot \\
			\llbracket \neg t \rrbracket 
			&\cong \neg \llbracket t 
			\rrbracket\\
			\llbracket t \land u 
			\rrbracket &\cong \llbracket t 
			\rrbracket \land \llbracket u
			\rrbracket\\
			\llbracket t = u \rrbracket 
			&\cong \llbracket t 
			\rrbracket = \llbracket u
			\rrbracket\\
			\llbracket c^{\tau_1 \to ... 
			\to \tau_n \to \texttt{bool}} 
			t_1 ... t_n \rrbracket &\cong 
			c^{\llbracket \tau_1 \rrbracket, 
			..., \llbracket \tau_n \rrbracket}
			(\llbracket t_1 \rrbracket, ..., 
			\llbracket t_n \rrbracket)\\
			\llbracket \forall x^{\tau}.t 
			\rrbracket &\cong (\forall 
			x^{\llbracket \tau \rrbracket}.
			\llbracket t \rrbracket)
		\end{align*}
	\item Variables and function 
		applications (of return type 
		other than \texttt{bool}), 
		including non-functional 
		constants have MSFOL 
		equivalents.
		\begin{align*}
			\llbracket x^{\tau} 
			\rrbracket &\cong 
			x^{\llbracket \tau \rrbracket}\\
			\llbracket c^{\tau_1 \to ... 
			\to \tau_n \to \tau} 
			t_1 ... t_n \rrbracket &\cong 
			c^{(\llbracket \tau_1 \rrbracket, 
			..., \llbracket \tau_n \rrbracket)
			\to \llbracket \tau \rrbracket}
			(\llbracket t_1 \rrbracket, ..., 
			\llbracket t_n \rrbracket)
		\end{align*}
	\end{itemize}

	The more interesting parts of the 
	transformation deal with 
	translating the non-first-order
	elements of HOL to MSFOL. These 
	include:
	\begin{itemize}
		\item Type variables $\alpha$
		and \textit{compount types} ---
		application type constructors
		$\kappa^n$ with $n > 0$.
		\item $\lambda$-abstractions 
		such as $\lambda x. t$.
		\item Variables of functional 
		types and partial applications 
		of functions to arguments are 
		possible in HOL but not in 
		MSFOL. For example, 
		$t^{\tau_1 \to \tau_2 \to 
		\tau}\ t_1$ is a 
		partial application of $t$ 
		which is a variable of type 
		$\tau_1 \to \tau_2 \to \tau$
		to argument $t_1$ and this 
		application has type 
		$\tau_2 \to \tau$.
		This is not directly 
		expressible in MSFOL.
	\end{itemize}

	\subsubsection{Monomorphization}
		Recall that HOL types $\tau$ are 
		either type	variables $\alpha$ or 
		applications of type constructors 
		$\kappa^n$ to $n$ types.
		A \textit{monomorphic type} is a
		type without type variables 
		(e.g. $\kappa^0$, $\kappa^1 
		\kappa^0$, and $\texttt{bool}^0 
		\to \texttt{int}^0$ 
		where $\to$ is 
		a type consructor), and 
		a \textit{schematic type} is one 
		with type variables (such as 
		$\alpha \to \texttt{bool}^0$). 
		Monomorphization involves 
		repeatedly instantiating schematic
		terms based on a set of 
		monomorphic terms until a fixed 
		point is reached.
		
		The definition of monomorphization 
		requires a description of 
		\textit{instantiation} of schematic 
		entities w.r.t. 
		monomorphic ones.
		\begin{itemize}
		\item Informally, a monomorphic type 
			$\tau_M$ \textit{matches} a schematic 
			type $\tau_S$ if it can replace the 
			type variables in the schematic 
			type. For example, \texttt{bool} 
			matches $\alpha$; and $\texttt{bool} 
			\to \texttt{int}$ matches 
			$\texttt{bool} \to \beta$
			only if \texttt{int} matches 
			$\beta$. Finally, for our running 
			example taken from \cite{bohme}, 
			${\texttt{bool} \to \kappa}$
			matches $\alpha \to \beta$,
			where $\kappa$ is a monomorphic
			type. 
			
		\item Given a monomorphic constant 
			$c^{\tau_M}$,  if $t^S$ is a 
			schematic term that contains a 
			schematic constant $c^{\tau_S}$ 
			such that $\tau_M$ matches 
			$\tau_S$, then $c^{\tau_M}$ induces 
			a substitution $\sigma$ on $t^S$, 
			and $\sigma(t^S)$ is an instance of 
			$t^S$ w.r.t. to $c^{\tau_M}$. 
			For example, the instance of term
			\begin{center}
				$(\forall f^{\alpha \to \beta},\ 
				x^{\alpha},\ xs^{\texttt{list }
				\alpha}.\ \texttt{apphd }f\ 
				(\texttt{cons }x\ xs) = f\ x)$
			\end{center} 
			w.r.t. constant
			\begin{center}
				$(\lambda x^{\texttt{bool}}.
				\texttt{if }x\texttt{ then }
				a^{\kappa} \texttt{ else }
				b^{\kappa})^{\texttt{bool} \to 
				\kappa}$ 
			\end{center}
			is
			\begin{center}
				$(\forall f^{\texttt{bool} 
				\to \kappa},\ x^{\alpha},\ 
				xs^{\texttt{list }\alpha}. 
				\texttt{apphd }f\ (\texttt{cons }
				x \ xs) = f\ x)$
			\end{center}
			where implicitly, the instantiation 
			has turned all occurrences of $f$ 
			in the term from $f^{\alpha \to \beta}$ 
			to $f^{\texttt{bool} \to \kappa}$.
		\item The idea of an instance is 
			extended to terms as follows. If 
			$t^M$ is a monomorphic term, then 
			$\sigma(t^S)$ is an instance of 
			$t^S$ w.r.t. $t^M$, if 
			$\sigma$, the combination of all 
			substitutions induced by constants 
			in $t^M$ is defined. This instance 
			can still be schematic. For example, 
			since $x^{\texttt{bool}}$ is an 
			instance of $x^{\alpha}$ w.r.t.
			$\texttt{T}^{\texttt{bool}}$, 
			and $xs^{\texttt{list bool}}$ is 
			an instance of 
			$xs^{\texttt{list }\alpha}$ 
			w.r.t. $[\ ]^{\texttt{list bool}}$,
			we can combine the instantiation 
			of the constant in the previous 
			example to obtain that the 
			instantiation of term
			\begin{center}
				$(\forall f^{\alpha \to \beta},\ 
				x^{\alpha},\ xs^{\texttt{list }
				\alpha}.\ \texttt{apphd }f\ 
				(\texttt{cons }x\ xs) = f\ x)$
			\end{center}
			w.r.t. the term
			\begin{center}
				$(\texttt{apphd }(\lambda 
				x.\ \texttt{if }x\texttt{ then }
				a \texttt{ else } b)^{\texttt{bool} 
				\to \kappa}\ (\texttt{cons T}\ 
				[\ ])) \neq a$
			\end{center}
			to be
			\begin{center}
				$(\forall f^{\texttt{bool}
				\to \kappa},\ x^{\texttt{bool}},
				\ xs^{\texttt{list bool}}.\ 
				\texttt{apphd }f\ (\texttt{cons }x
				\ xs) = f\ x)$
			\end{center}
		\item Instantiation can further be 
			extended to sets of schematic 
			terms $S$ and monomorphic terms $M$. 
			An instance of $S$ w.r.t. 
			$M$ is the set $I$ of terms such 
			that each term in $I$ is an instance 
			of some term from $S$ w.r.t. 
			some term from $M$. Since 
			instantiation can produce either 
			monomorphic or schematic terms, $I$
			can be partitioned into either as
			$(I_M, I_S)$.
		\end{itemize}
		Now, we can describe monomorphization.
		\begin{itemize}
		\item A \textit{monomorphization 
			step} for $S$ w.r.t. $M$ maps 
			the pair $(M,S)$ to the pair 
			$(M \cup I_M, S \cup I_S)$.
		\item The complete 
			\textit{monomorphization} of $S$ 
			w.r.t. $M$ is the computation of 
			a least fixed point of 
			monomorphization steps of $S$ 
			w.r.t. $M$.
		\item Given a HOL formula $F$, 
			if $(M, S)$ is the partition of its 
			constituents into monomorphic and 
			schematic terms, then 
			monomorphization of $S$ 
			w.r.t. $M$ yields pair 
			$(M^{\prime}, S^{\prime})$ and we 
			call the conjunction of all terms in 
			$M^{\prime}$ the monomorphization 
			of $F$.
		\end{itemize}
		Thus, using the above examples as 
		instantiation steps, we have that 
		monomorphization translates the formula $F$
		\begin{align*}
			F:\ &(\forall f^{\alpha \to \beta},\ 
			x^{\alpha},\ xs^{\texttt{list }\alpha}.\ 
			\texttt{apphd }f\ (\texttt{cons }x
			\ xs) = f\ x)\ \land\ \\
			&((\texttt{apphd }(\lambda x.\ 
			\texttt{if }x \texttt{ then }a 
			\texttt{ else } b)^{\texttt{bool} 
			\to \kappa}\ (\texttt{cons T}\ [\ ])) 
			\neq a)
		\end{align*}
		to the formula $F^{\prime}$
		\begin{align*}
			F^{\prime}:\ &(\forall 
			f^{\texttt{bool} \to \kappa},\ 
			x^{\texttt{bool}},\ 
			xs^{\texttt{list bool}}.\ 
			\texttt{apphd }f\ (\texttt{cons }x
			\ xs) = f\ x) \ \land\ \\
			&((\texttt{apphd } (\lambda x.\ 
			\texttt{if }x \texttt{ then }a 
			\texttt{ else } b)^{\texttt{bool} 
			\to \kappa}\ (\texttt{cons T}\ 
			[\ ])) \neq a).
		\end{align*}
		\noindent The following are some 
		limitations of this process:
		\begin{itemize}
		\item There have to be monomorphized 
			term in a formula, to guide the 
			monomorphization process, so it 
			seems like this process would 
			fail with formulas that contain 
			only schematic terms. 
		\item The monomorphization process 
			could be non-terminating. In 
			other words, the first component
			of the pair yielded by 
			monomorphization --- the 
			set of monomorphic terms ---
			could be infinite. For example,
			consider $S = \{c^{\alpha}
			\land c^{\kappa\ \alpha}\}$
			and $M = \{c^{\kappa_0}\}$.
			Now, $\kappa_0$ matches 
			$\alpha$, so the instance
			$c^{\kappa_0} \land 
			c^{\kappa\ \kappa_0}$ is added 
			to $M$ after a monomorphization 
			step. Now, $\kappa\ \kappa_0$
			matches $\alpha$, so the 
			instance $c^{\kappa\ \kappa_0} 
			\land c^{\kappa\ \kappa\ 
			\kappa_0}$ is added to $M$, 
			and so on. Since the proof of a 
			formula, if it existed, would 
			be finite, most monomorphic 
			terms would be irrelevant. 
			Finding the finite subset of 
			necessary monomorphic terms is 
			undecidable~\cite{10.1007/978-3-642-24364-6_7},
			but \cite{bohme} use heuristic
			methods to overapproximate
			this set. They limit the 
			number of monomorphization 
			steps and the number of 
			monomorphic terms generated
			with the expectation that 
			monomorphic terms that 
			contribute to proofs 
			are typically generated early 
			in the monomorphization process.
		\item The monomorphization process
			is described as a syntactic 
			process. Semantic steps in 
			the process aren't described, 
			and if it doesn't indeed 
			involve	any semantic pruning of 
			translations, the number of 
			monomorphization steps 
			might be impractically large. 
			For example, from the formula 
			in the running example above ($F$), 
			$x^{\alpha}$ could match 
			$a^{\kappa}$ and be instantiated 
			to $x^{\kappa}$. While 
			syntactically, this would 
			check out, semantically, 
			given that $f^{\alpha \to \beta}$
			is instantiated to 
			$f^{\texttt{bool} \to \kappa}$, 
			and that $f$ is applied to $x$, 
			$x$ has to have type 
			\texttt{bool} and this can be 
			achieved by matching it with 
			$x^{\texttt{bool}}$.
		\item Monomorphization is 
			incomplete and the instantiations 
			are done heuristically. This 
			means that if the SMT solver 
			is not able to prove the 
			monomorphized version of a 
			formula, a different 
			monomorphization could 
			very well be provable. Thus, 
			the monomorphization of a 
			problem would have to be 
			done smartly, and sometimes
			repeated multiple times to 
			be successful. 
		\end{itemize}
	
	\subsubsection{Lambda-Lifting}
		$\lambda$-abstractions represent 
		anonymous or unnamed functions, 
		which aren't allowed in MSFOL.
		These abstractions are removed
		from HOL formulas by a process 
		called \textit{$\lambda$-lifting}
		which uses a fresh constant as a
		name for the abstraction and adds 
		a quantified formula specifying 
		it's behavior. Concretely,
		\begin{align*}
			\llbracket t[\lambda x^{\tau}.u]
			\rrbracket \cong 
			(t[(\lambda x^{\tau}.u) \mapsto c]
			\land (\forall x^{\tau}.\ c\ x = u))
		\end{align*}
		The notation $t[x]$ represents a 
		term $t$ with a sub-term $x$, 
		and $t[x \mapsto y]$ is the 
		term obtained by substituting all 
		occurrences of $x$ by $y$ in $t$.
		$c$ is specified in MSFOL as an 
		uninterpreted function with sort 
		$\tau \to \kappa$ where $\kappa$ 
		is the sort of $u$. For instance, 
		$(\lambda x^{\texttt{int}}, x + 1)\ 
		5 = 6$
		is translated to $(c\ 5 = 6) \land
		(\forall x^{\texttt{int}}.\ 
		c\ x = x + 1)$.
		%If the 
		%$\lambda-$ abstraction contains free
		%variables, then they are turned 
		%into additional arguments to $c$.
		
	\subsubsection{Explicit Applications}
		For functional constants occurring 
		with a variable number of arguments, 
		\cite{bohme}, the minimal 
		number of arguments are 
		considered as the arity of the 
		function, and any additional 
		arguments are expressed 
		explicitly, with the help of a 
		constructor 
		$\texttt{app}^{(\alpha_1 \to 
		\alpha_2) \to \alpha_1 \to 
		\alpha_2}$ that is defined as:
		\begin{center}
			$\texttt{app }t_1\ t_2 = 
			t_1\ t_2$
		\end{center}
		If $c$ is a constant that 
		occurs with at least $n$
		arguments, then 
		occurrences of applications 
		of $c$ are translated as:
		\begin{center}
			$\llbracket c\ t_1\ ...\ t_n
			\ u_1\ ...\ u_m \rrbracket
			\cong \texttt{app}\ (...\ 
			(\texttt{app }(c\ t_1\ 
			...\ t_n)\ u_1)	\ ...)\ u_m$
		\end{center}
		where $m$ could be $0$, in which 
		case there are no applications 
		of \texttt{app}. For example, if 
		$f^{\texttt{int} \to \texttt{int}}$ 
		occurs twice in a formula, once 
		partially applied to no arguments 
		as $f$, and once fully applied as 
		$(f\ 0)$, then since the minimal 
		number of arguments to $f$ is 
		zero, we have:
		\begin{align*}
		\llbracket f \rrbracket &\cong
		f\\
		\llbracket f\ 0 \rrbracket &\cong
		\texttt{app } f\ 0
		\end{align*}
		
		\texttt{app} is a higher-order 
		function so both the function 
		($f$ in the above example) and 
		\texttt{app} have to be encoded 
		in the SMT solver as constructors.
		This can also be done, for example, 
		with arrays which are essentially 
		functional types which can be passed
		around, since arrays in SMTLIB 
		can have any index and value types, 
		and map elements of the index type 
		to the value type. Bohme et al. 
		don't specify what technique 
		they use to encode the 
		\texttt{app} constructor.
		
		Using the \texttt{app} constructor
		for partially applied interpreted 
		constants in MSFOL, such as 
		logical connectives, would result 
		in terms that aren't well-typed.
		These are $\eta-$expanded and 
		then $\lambda-$lifted. For example, 
		for the partially applied 
		conjunction ($\land$, used as prefix 
		here), a partial application is 
		translated as follows.
		\begin{align*}
			\llbracket t [\land\ x] 
			\rrbracket &\cong \llbracket 
			 t[\lambda y.\ \land\ x\ y]
			\rrbracket\\
			&\cong t[c]  \land (\forall y.\ 
			c\ y = \land\ x\ y)
		\end{align*}
		
	\subsubsection{Erasure of Compound Types}
		A type constructor $\kappa^n$ with 
		$n > 0$ is applied to $n$ types 
		$\tau_i$. After monomorphization, 
		each $\tau_i$ is monomorphic, so 
		$\kappa^n\ \tau_1\ ...\ \tau_n$ 
		is a monomorphic compound type. It
		is represented as a fresh nullary 
		type constructor $\kappa_0^n$ with 
		the same interpretation as 
		$\kappa^n\ \tau_1\ ...\ \tau_n$, 
		which can now be represented in 
		MSFOL.
		\begin{center}
			$\llbracket \kappa^n\ 
			\tau_1\ ...\ \tau_n \rrbracket
			\cong \kappa_0^n$
		\end{center}
		Some examples are:
		\begin{align*}
			\llbracket \texttt{bool} \to
			\kappa \rrbracket &\cong \kappa_1\\
			\llbracket \texttt{list\ bool}
			\rrbracket &\cong \kappa_2
		\end{align*}
		Many SMT solvers now support sorts 
		parameterized by other sorts and 
		algebraic data types which can be 
		used to encode compound types.
		
		While this is a straightforward
		translation step syntactically, 
		notice that SMT solvers need 
		extra help via premise selection 
		to solve constraints involving 
		these types. Integers are 
		axiomatized in the SMT solver as 
		the theories of linear and 
		non-linear integer arithmetic, and 
		Isabelle/HOL integers are made to 
		correspond to these in the translation, 
		so when a formula containing 
		integers is sent to the SMT solver, 
		it can perform its own reasoning. 
		However, SMT solvers have no 
		axiomatization of lists of, say, 
		Booleans. It is crucial that when 
		\texttt{list bool} is 
		translated to $\kappa_2$, the 
		premise selector of sledgehammer
		selects the right lemmas about 
		Boolean lists to give to the 
		SMT solver so it can reason 
		about them.
		 
\bibliographystyle{abbrv}
\bibliography{bib2}

\end{document}