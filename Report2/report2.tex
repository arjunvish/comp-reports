\documentclass{article}
\usepackage{proof}
\usepackage{bussproofs}
\usepackage{xcolor}
\usepackage{url}
\usepackage{framed}
\usepackage{amsmath, amsfonts, amssymb}
\usepackage{mathtools}
\usepackage{minted}

\usepackage{hyperref}
\hypersetup{
 pdfborder={0 0 0},
 colorlinks=true,
 linkcolor=blue,
 urlcolor=blue,
 citecolor=blue
}
\usepackage{cleveref}


\newcommand{\rem}[1]{\textcolor{red}{[#1]}}
\newcommand{\ed}[1]{\textcolor{blue}{#1}}
\newcommand{\ct}[1]{\rem{#1 --ct}}

\begin{document}
\title{Comprehensive Exam Report 2 - Relating Higher Order and First-Order Logics}
\author{Arjun Viswanathan}
\date{}
\maketitle

\section{Introduction}
\label{sec:intro}
	In this report, we will discuss the 
	internals of two tools that use ATPs to 
	provide automation to ITPs - 
	SMTCoq~\cite{DBLP:phd/hal/Keller13} and 
	Sledgehammer's SMT solver 
	integration~\cite{bohme}. Specifically, 
	we will look at the following aspects 
	of each tool. For SMTCoq, we will study 
	how proof by reflection is used to 
	\textit{reflect} proofs from an SMT solver
	to Coq's logic, and the data structures 
	used to facilitate this reflection.
	Sledgehammer provides a translation of 
	a fragment higher-order logic into 
	first-order logic, so that 
	SMT-solvers can reason about them. We 
	will look into this translation in 
	detail.

\section{SMTCoq}
\label{sec:smtcoq}
	SMTCoq is a skeptical cooperation between 
	the Coq proof assistant, and SAT and SMT 
	solvers that was implemented as a Coq 
	plugin.
	
	Automated theorem provers (ATPs) like SAT 
	and SMT solvers are susceptible to bugs 
	due to the large code-bases used to 
	support	their automaticity. Interactive 
	theorem provers (ITPs) like Coq have a 
	small trustable proof kernel which is 
	liable to be compromised if it trusts 
	external results. To avoid extending 
	its trust-base, SMTCoq requires the 
	ATPs to be proof-producing, and uses 
	Coq's computational capabilities to 
	check these, in a process called 
	computational reflection. 
	
	\subsection{Proof by Reflection}
	\label{reflect}
	Computational reflection is used 
	to prove propositions, or formulas 
	of type \texttt{Prop} in Coq,
	using external ATPs that can prove 
	these formulas. To avoid 
	compromising Coq's kernel, the
	external ATP doesn't directly 
	produce proofs for \texttt{Prop}s.
	Instead, formulas in the ATP's 
	language are represented in Coq
	using data structures. This is 
	called a \textit{deep embedding}.
	Then, Coq formulas, represented 
	using Coq types such as 
	\texttt{Prop} form a 
	\textit{shallow embedding}. 
	A proof produced by an ATP in 
	the deep embedding is 
	\textit{reflected} to a proof 
	of a Coq \texttt{Prop} (the 
	shallow embedding). This 
	process requires an 
	\textit{interpretation function}
	which defines the meaning of 
	formulas in the deep embedding in 
	terms of those in the shallow 
	embedding. Additionally, the 
	\textit{reflection principle} is 
	a generic lemma that relates 
	formulas in the two embeddings. 
	This is a powerful lemma that 
	relates all formulas represented
	as data structures in Coq, to 
	formulas represented as 
	\texttt{Prop}s. The proof 
	involves an induction 
	on the structure of the data 
	structures in order to prove 
	their similarity to the shallow 
	terms, while using the
	interpretation function. This 
	proof is factored once and 
	can be reused by instantiating 
	it with specific formulas 
	proved by the external solvers. 
	Work is then done by Coq's 
	computational mechanism to 
	check that the proof trace
	from the SMT solver proves 
	the formula it claims to prove.
	
	
\bibliographystyle{abbrv}
\bibliography{bib2}

\end{document}