\documentclass{article}
\usepackage{proof}
\usepackage{bussproofs}
\usepackage{xcolor}
\usepackage{url}
\usepackage{framed}
\usepackage{amsmath, amsfonts, amssymb}
\usepackage{mathtools}
\usepackage{minted}
\usepackage{stmaryrd}
\usepackage{verbatim}

\usepackage{hyperref}
\hypersetup{
 pdfborder={0 0 0},
 colorlinks=true,
 linkcolor=blue,
 urlcolor=blue,
 citecolor=blue
}
\usepackage{cleveref}


\newcommand{\rem}[1]{\textcolor{red}{[#1]}}
\newcommand{\ed}[1]{\textcolor{blue}{#1}}
\newcommand{\ct}[1]{\rem{#1 --ct}}

\begin{document}
\title{Comprehensive Exam Report 2 - Relating Higher Order and First-Order Logics}
\author{Arjun Viswanathan}
\date{}
\maketitle

\section{Introduction}
\label{sec:intro}
	In this report, we will discuss the 
	internals of two tools that use 
	automatic theorem provers (ATPs) to 
	provide automation to interactive 
	theorem provers (ITPs) - 
	SMTCoq~\cite{DBLP:phd/hal/Keller13} 
	and Sledgehammer's SMT solver 
	integration by Bohme et al.~\cite{bohme}. 
	Specifically, we will look at the 
	following aspects of each tool. For 
	SMTCoq, we will study 
	how proof by reflection is used to 
	\textit{reflect} proofs from an SMT 
	solver to Coq's logic, and the data 
	structures used to facilitate this 
	reflection.	Sledgehammer provides a 
	translation from higher-order logic 
	(HOL) to many-sorted first-order logic 
	(MSFOL), so that SMT-solvers can be 
	used on HOL problems. We will look at 
	this translation in detail.

\section{SMTCoq}
\label{sec:smtcoq}
	SMTCoq is a skeptical cooperation 
	between the Coq proof assistant, and 
	SAT and SMT solvers, implemented as a 
	Coq plugin. We will maintain a focus 
	on the SMT solver integration of 
	SMTCoq, noting that most features are 
	shared with the SAT	solver integration.
	
	ATPs like SMT solvers are susceptible 
	to bugs due to the large code-bases 
	used to support	their automation. 
	ITPs like Coq have a small trustable 
	proof kernel which would be 
	compromised if they were to trust
	external results. In a collaboration
	with SMT solvers, to avoid extending 
	Coq's trust-base, SMTCoq requires the 
	solvers to be proof-producing, and uses 
	Coq's computational capabilities 
	to lift these up to Coq proofs, in a 
	process called computational 
	reflection.
	
	\subsection{Proof by Reflection}
	\label{reflect}
	\texttt{Prop} is the type of 
	constructive propositions in Coq. 
	Additionally, Coq also has a 
	classical Boolean type, 
	\texttt{bool}, which is 
	computational --- it is an 
	inductive type with constructors 
	\texttt{true} and \texttt{false}. 
	As a consequence, one can 
	perform case analysis on 
	\texttt{bool} values since they 
	are decidable. Propositions 
	aren't decidable - one cannot 
	observe whether any given 
	proposition is true or false. 	
	Using the \texttt{bool} type, a 
	first-order formula can be 
	expressed in Coq as a \texttt{bool}
	term. For example, for Booleans 
	\texttt{a} and \texttt{b}, 
	\texttt{implb a (a || b)} 
	expresses the first-order formula 
	$a \to a \lor b$, where the logical 
	implication and logical disjunction 
	symbols are \texttt{implb} and 
	\texttt{||} for Coq Booleans, and 
	those for first-order logic are 
	$\to$ and $\lor$, respectively. The 
	fact that a Boolean formula in Coq 
	holds is expressed by equating it to 
	true.
\begin{minted}{Coq}
Lemma BoolForm : forall (a b : bool), implb a (a || b) = true.
\end{minted}
	Note that this is a classical formula 
	in Coq expressed using the Boolean type. 
	Using Coq's constructive proposition type,
	the same formula is expressed as follows.
\begin{minted}{Coq}
Lemma PropForm : forall (a b : Prop), a -> a \/ b.
\end{minted}
	where \texttt{->} and $\backslash/$ are 
	Coq's propositional implication and 
	disjunction. SMTCoq proves formulas 
	in Coq that have Boolean type, and 
	not formulas over Coq propositions.
	 
	The process of computational 
	reflection relies heavily on 
	Coq's \textit{conversion rule}.
	Coq's Calculus of Inductive 
	Constructions (CIC) is a 
	$\lambda-$calculus that has a 
	reduction mechanism for terms. The
	reduction rules form a strongly 
	normalizing system. The calculus's
	conversion rule allows a term to 
	have multiple types, as long as the 
	types have the same normal forms. For 
	instance, for some predicate $P$ 
	over natural numbers, a term that 
	has type $P(10)$ also has type 
	$P(5*2)$ and $P(20-10)$. Due 
	to the conversion rule, 
	computations can be used in Coq's 
	reasoning and proof terms can be 
	found simply by computing normal 
	forms of types.
	
	\begin{comment}
	The 
	conversion rule includes more 
	complicated reductions of 
	types that include $\beta$-reductions,
	$\alpha-$conversions (that perform
	variable renaming), 
	$\zeta-$conversions (of local 
	\texttt{let} definitions), and 
	$\eta-$conversion of terms with a 
	functional type. These are explained 
	using the following example: the type 
	of the term 
	\begin{center}
		\texttt{(fun (x : Type $\to$ Type) 
		$\implies$ x) (fun (z : Type) 
		$\implies$ let a := z in a)}  
	\end{center}
	is the same as the type of the term 
	\begin{center}
		\texttt{fun (y : Type) $\implies$ y}.
	\end{center}
	In other words, the two terms are 
	convertible to each other. 
	\texttt{Type} is part of Coq's
	type hierarchy, and an instance of 
	\texttt{Type} is a set of terms 
	(or a type at the bottom of the 
	hierarchy) such as \texttt{nat},
	the type of natural numbers (to be 
	precise, \texttt{Set} is the 
	type of \texttt{nat} and 
	\texttt{Type} is the type of 
	\texttt{Set}, but this simplification
	still makes the point). The syntax
	\texttt{fun (x : $\alpha$) $\implies$
	y} represents a $\lambda$-abstraction 
	$\lambda(x : \alpha).y$. The 
	type conversion takes place as follows.
	The term
	\begin{center}
		\texttt{(fun (x : Type $\to$ Type) 
			$\implies$ x) (fun (z : Type) 
			$\implies$ let a := z in a)}  
	\end{center}
	reduces to
	\begin{center}
	\texttt{(fun (x : Type $\to$ Type) 
		$\implies$ x) (fun (z : Type) 
		$\implies$ z)}	
	\end{center}
	by $\zeta-$reduction of the \texttt{let}
	expression. $\beta-$reducing this 
	term, gives 
	\begin{center}
		\texttt{fun (z : Type) 
			$\implies$ z}	
	\end{center}	 
	which is $\alpha-$convertible to 
	\begin{center}
		\texttt{fun (y : Type) $\implies$ y}.
	\end{center}
	by renaming the $z$ to $y$.
	\end{comment}
	
	A proposition in Coq is proved by 
	providing to Coq's type checker, a 
	proof term that inhabits the 
	proposition; in other words, a 
	proof term that has the proposition 
	as its type. Automation can be 
	added to Coq by implementing 
	methods in Coq, that find the right 
	proof term for propositions --- a 
	process called \textit{proof search}. 
	Such a method could use an external 
	solver, such as an ATP, to guide 
	proof search. However, ATPs 
	typically provide automatic 
	proofs without justification - 
	by answering the question `does F 
	hold?' with	yes/no. To use the 
	result of an SMT solver without 
	compromising Coq's kernel, the ATP 
	would have to produce, in addition 
	to a result, a proof object
	that justifies the result. This 
	proof object will guide the proof search 
	in Coq in a process called 
	\textit{proof reconstruction}. 
	Reconstruction can be implemented 
	as a meta-procedure. For example, 
	reconstruction of SMT solver proofs 
	in Isabelle/HOL are done external to 
	the prover's logic~\cite{bohme}. 
	This reconstruction can also be 
	expressed within the prover's logic 
	by a process called 
	\textit{computational reflection}, 
	which relies on the previously 
	mentioned reduction mechanism and 
	the conversion rule.
	
	Given a logical framework A, and 
	a logical framework B, in which we 
	want to represent the language of A, 
	there are two ways to achieve this
	representation, and 
	both of these are necessary for 
	reflection:
	\begin{itemize}
		\item \textit{Shallow Embedding: }
		Use types and terms in B that 
		correspond to those of A, to 
		represent terms of A.
		\item\textit{Deep Embedding: }
		Define custom types in B that 
		correspond to those of A, and 
		define inside B, the meaning of 
		terms of these custom types with 
		respect to shallow terms.
	\end{itemize}

	SMT solvers prove formulas over 
	sorted (typed) terms and these 
	sorts correspond to types in 
	Coq that are specified in the 
	Coq standard library. Terms of 
	these library types represent a 
	shallow embedding of SMT 
	terms within Coq and SMTCoq is used 
	to prove Coq propositions over such 
	shallow terms. Examples are the 
	Coq library for integers or 
	the $\mathbb{Z}$ type~\cite{CoqZ},
	and the one for Booleans or the 
	\texttt{bool} type~\cite{CoqBool}.
	The deep embedding of SMT terms 
	in Coq uses Coq's inductive types,
	machine integers and arrays. In 
	addition to the other benefits 
	of reflection mentioned below,
	the choice of machine integers 
	and arrays contribute to 
	efficient storage of large 
	SMT proofs. Deep terms are harder 
	to specify, since features such as 
	binders 
	and variable substitution must be 
	handled by writing functions 
	for these within Coq. These 
	features are available as part 
	of Coq's standard library 
	for shallow terms. 
	
	The main components of SMTCoq are 
	a \textit{checker}, that checks 
	a proof produced by an ATP for a 
	formula against the deep embedding 
	of that formula; and a 
	\textit{reflection principle}, 
	or a correctness lemma of the checker,
	that performs the reflection of the 
	formula	in the deep embedding to the 
	formula in the shallow embedding. 
	An \textit{interpretation function}
	defines the meaning of terms and types 
	in the deep embedding in terms of their
	shallow counterparts and is used in the 
	reflection principle. The reflection 
	principle 
	proves that if a checker is able to 
	check a formula in the deep embedding,
	its interpretation (in the shallow 
	embedding) is true. A proof of 
	a formula in the shallow embedding 
	in Coq is then just an instantiation 
	of this lemma with a deep formula that 
	the SMT solver is able to prove.
	Work is then done by Coq's 
	computational mechanism in running
	the checker on the deep embedding 
	of the formula and the proof 
	from the SMT solver. Deep and shallow 
	embeddings were first introduced 
	while discussing embedding 
	hardware description languages 
	in the HOL theorem 
	prover~\cite{10.5555/645902.672777}.
	
	Formally, reflection of proofs 
	in SMTCoq is defined as follows. 
	In Coq, we have 
	\begin{itemize}
		\item \texttt{form} --- a data 
		structure representing the deep 
		embedding of SMT formulas
		\item \texttt{bool} --- the 
		Boolean type in Coq 
		representing the shallow 
		embedding of SMT formulas
		\item $\mathbb{I}$ --- the 
		interpretation function, which 
		is a Boolean predicate over 
		\texttt{form}s and 
		valuations of free variables 
		(of type 
		\texttt{form} $\to$ \texttt{val}
		$\to$ \texttt{bool}). The interpretation
		function provides a fixed 
		interpretation of types, and 
		terms modulo free variables. The 
		interpretation
		of terms uses a valuation function
		$\nu$ that interprets 
		free variables. The interpretation
		of a formula \texttt{f} for a 
		valuation $\nu$ is specified as 
		$\mathbb{I}_{\nu}(\texttt{f})$.
		\item \texttt{T} --- the type of 
		proof objects from the SMT solver
	\end{itemize}
	Reflection requires the following:
	\begin{itemize}
		\item a function 
		\texttt{checker : form $\to$ T $\to$ 
		bool} implemented in Coq such that 
		\texttt{checker}$\ f\ t$ is 
		\texttt{true} if proof object $t$
		justifies $f$
		\item the reflection principle, that 
		proves the correctness of 
		\texttt{checker}:
		\begin{align*}
			\texttt{checker\_correct :} 
			\forall\ (f : \texttt{form})\ 
			(t : \texttt{T})\ 
			\texttt{checker}\ f\ t\ = 
			\texttt{true} \to \\
			\forall\ (\nu : \texttt{val}),\ 
			\mathbb{I}_{\nu}(f) = 
			\texttt{true}
		\end{align*}
	\end{itemize}
	The reflection principle allows
	us to use \texttt{checker}, which 
	is computationally heavy, to 
	produce proofs of formulas. For any
	formula $f$, and SMT proof $t$ of $f$,
	the proof of $\forall\ \nu,\ 
	\mathbb{I}_{\nu}(f) = 
	\texttt{true}$ is simply:
	\begin{center}
		$\texttt{checker\_correct}\ \ 
		f\ \ t\ \ 
		(\texttt{refl\_equal}\ \ 
		(\texttt{checker}\ f\ t)\ \ 
		\texttt{true})$
	\end{center}
	where \texttt{refl\_equal} 
	(reflexivity of equality) is a tactic
	that forces the Coq type checker to 
	perform an equality check between 
	$\texttt{checker}\ f\ t$ and \texttt{true}
	by reduction. The only difference 
	in each invocation of the reflection 
	principle is the checking 
	done by \texttt{checker}. The 
	\texttt{refl\_equal} calls the 
	checker and puts the burden of the 
	proof on Coq's reduction mechanism
	to reduce $\texttt{checker}\ f\ t$
	to \texttt{true}. Since proofs from 
	SMT solvers can be quite large, their
	reduction by \texttt{checker} may
	include non-trivial computations.
	
	\subsection{Deep and Shallow Embeddings}
	In the following, we present 
	a simplified version of SMTCoq's 
	deep embedding. The actual embedding 
	uses Coq's machine integers to 
	implement sharing of terms and 
	optimize space, but an unoptimized 
	and simpler version is presented 
	below. 
	
	\texttt{form} is the type of 
	formulas:
	\begin{minted}{coq}
Inductive form : Type :=
	| Fatom (_ : atom)
	| Ftrue
	| Ffalse
	| Fnot (_ : form)
	| Fand (_ : array form)
	| For (_ : array form)
	| Fimp (_ : array form)
	| Fxor (_ _ : form)
	| Fiff (_ _ : form)
	| Fite (_ _ _ : form)
	\end{minted}
	where a formula can be an atom, 
	\texttt{true}, \texttt{false};
	a negation of a formula; a 
	conjunction, disjunction, or 
	implication of two or more 
	formulas; an exclusive-or or
	an equivalence of two formulas; 
	or an if-then-else of three formulas.
	\texttt{atom} is the type of atoms, 
	which can be variables, 
	constants, or applications of 
	functions belonging to 
	different theories.
	
	SMTCoq implements the \texttt{checker}
	function as the \textit{main checker}
	which composes the work done 
	by various \textit{small checkers}. 
	The main checker has the following 
	type.
	\begin{center}
		\texttt{main\_checker ($input$ : 
			form) ($proof$ : T) : bool}	
	\end{center}
	It checks whether proof object
	$proof$ from the SMT solver 
	is a proof of the deep embedding 
	$input$, and returns \texttt{true} if 
	it is. Specifically, \texttt{main\_checker}
	checks whether $proof$ is a proof of 
	the unsatisfiability of the negation 
	of $input$, i.e., $\neg input$. This is 
	equivalent to a proof of validity of 
	$input$. The checking of unsatisfiability 
	is divided into multiple \textit{steps}
	and each step is independently checked 
	by a small checker. Currently, there are 
	small checkers for steps that perform 
	resolution, conversion of formulas to 
	conjunction normal form (CNF), SMT 
	solver simplifications, and one for 
	each theory: equality over 
	uninterpreted functions (EUF), linear 
	integer arithmetic (LIA), bit-vectors 
	(BV), and functional arrays with 
	extensionality (AX). Each small
	checker simplifies the input 
	\texttt{form} via the step it performs,
	and the main checker ultimately checks
	that the input is finally transformed to 
	$\bot$ or \texttt{False} which is the 
	simplest unsatisfiable formula.
	The reflection principle is a proof of 
	correctness of the main checker:
	\begin{align*}
		\texttt{main\_checker\_correct: } 
		&\forall\ input\ proof,\   
		\texttt{main\_checker }input\  
		proof = \texttt{true}\\
		&\to \forall\ \nu,\ \mathbb{I}_{\nu}(f)
		= \texttt{true}
	\end{align*}
	where $\nu$ is the valuation 
	of free variables used by 
	the interpretation function 
	$\mathbb{I}$. If the checker 
	is successful in checking the 
	proof of a formula in the deep
	embedding, the reflection 
	principle converts this into 
	a proof of the formula in the 
	shallow embedding, via the 
	interpretation function, and 
	by doing this for all valuations,
	it proves the validity of the 
	formula (it is true for any 
	model). Since each small
	checker modifies $input$, 
	there is a similar correctness 
	proof for each small checker 
	that guarantees that they don't
	change the unsatisfiability of 
	$input$ during this process, and 
	\texttt{main\_checker\_correct}
	is composed of the correctness
	proofs of the small checkers.
	
	The checker is \textit{sound} ---
	when it returns \texttt{true} for 
	a formula, it is valid and its 
	negation is unsatisfiable --- but 
	not \textit{complete} --- when it 
	returns \texttt{false}, we can't be 
	certain that the formula isn't valid
	or that its negation is satisfiable.
	This is because, the underlying SMT 
	solver 
	cannot decide certain formulas -
	given a formula, it could either 
	return \texttt{sat} 
	(satisfiable), \texttt{unsat} 
	(unsatisfiable), or 
	\texttt{unknown} (`don't know').

	
\section{Sledgehammer}
\label{sec:hammer}
	Isabelle~\cite{DBLP:journals/corr/cs-LO-9301106} 
	is an LCF-style system that 
	provides a meta-logic which can be 
	instantiated with other logics.
	Isabelle/HOL~\cite{10.5555/1791547}, 
	one of the most popular Isabelle 
	instantiations, implements a 
	classical higher-order logic. 
	
	Sledgehammer is
	an Isabelle/HOL component that 
	uses external ATPs to enhance 
	Isabelle/HOL with proof 
	automation. Initially, these 
	ATPs only included resolution 
	provers~\cite{10.1007/978-3-642-39799-8_1}.
	The work by Bohme et 
	al.~\cite{bohme} involved 
	extending Sledgehammer to 
	incorporate SMT
	solvers~\cite{Barrett2018} and this 
	work will be our focus for the 
	rest of this section. As with 
	SMTCoq, the SMT solvers integrated
	with Sledgehammer produce a 
	proof object which is 
	then reconstructed within
	Isabelle/HOL by Sledgehammer, 
	using Sledgehammer's own internal 
	ATP --- Metis~\cite{hurd2003d}. The 
	proof object essentially guides 
	the inference steps of the proof 
	within Isabelle/HOL.
	
	Given a conjecture $\Phi$ in 
	Isabelle/HOL, Sledgehammer 
	selects a set of facts 
	$\Gamma$ that might be relevant 
	to proving $\Phi$ and sends
	the formula $\Gamma \land \neg 
	\Phi$ to the SMT solver to check 
	for unsatisfiability. If the SMT 
	solver is able to refute the 
	conjecture (i.e., conclude 
	the unsatisfiability of its 
	negation), it returns 
	a proof. Since Isabelle/HOL 
	implements a higher-order logic 
	(HOL), which 
	is more expressive than 
	the SMT solver's many-sorted
	first order logic (MSFOL),
	the entirety of the formula
	$\Gamma \land \neg \Phi$ may not 
	be understandable to the SMT 
	solver. Parts of the formula,
	however, may be fully 
	first-order (FOL is 
	a subset of HOL). Bohme et al.
	extended Sledgehammer with 
	a translation from higher-order 
	to first-order logic, so that 
	it can reason about the remainder
	of the formula. This translation
	is discussed in the rest of this 
	section.
	
	\subsection{Higher-Order Logic}
	\label{sec:hol}
	In this section, we specify the 
	syntax of higher-order logic 
	and delegate the explanation of 
	semantics to a 
	reference~\cite{10.5555/155278}. 
	HOL consists of 
	types $\tau$ and terms $t$. 
	\begin{align*}
	\tau &:= \alpha\ |\ \kappa^n\ 
	\tau_1 ... \tau_n\\
	t &:= x^{\tau}\ |\ c^{\tau}\ |\ t_1\ t_2\ 
	|\ \lambda x^{\tau}.t
	\end{align*}	
	Types $\tau$ are either type
	variables $\alpha$ or 
	applications of type 
	constructors $\kappa^n$ to 
	$n$ types ($n$ is usually omitted). 
	Particular types of interest are 
	the function type --- formed by 
	applying the arrow type constructor 
	$\to^{2}$ to other types, the 
	Boolean type --- $\texttt{bool}$ 
	(or $\texttt{bool}^0$), the type of 
	natural numbers --- \texttt{nat},
	and integers --- \texttt{int}.
	Terms are typed variables, 
	typed constants, applications 
	of terms to terms, or typed
	$\lambda-$ abstractions. We have
	the usual Boolean constants 
	representing logical connectives
	and quantifiers. For instance, 
	logical negation, 
	$\neg^{\texttt{bool} \to 
	\texttt{bool}}$, universal 
	quantification,
	$\forall^{\alpha \to 
	\texttt{bool} \to \texttt{bool}}$, 
	and polymorphic equality,
	$=^{\alpha \to \alpha 
	\to \texttt{bool}}$. Type 
	annotations for terms are also 
	often omitted when understood
	from context.

	\subsection{Many-Sorted First-Order Logic}
	\label{sec:msfol}
	Many-sorted first-order logic extends
	first-order logic (FOL) with 
	types or sorts. We present the 
	syntax in this section and the 
	semantics are presented in
	\cite{Barrett2018}. Syntactically, 
	the components of MSFOL are sorts 
	$\sigma$, terms $t$, and 
	formulas $\phi$. Sorts are 
	atomic entities that 
	represent types. Function types 
	--- ($\sigma_1$, ..., $\sigma_n$) 
	$\to$ $\sigma$ ---
	and relation types 
	--- ($\sigma_1$, ..., $\sigma_m$)
	are defined over sorts, and 
	are types of functions and 
	predicates below. Terms and 
	formulas are specified as:
	\begin{align*}
		t &:= x^{\sigma}\ |\ 
		f^{(\sigma_1, ..., \sigma_n) \to 
		\sigma}	(t_1, ..., t_n)\\
		\phi &:= \bot\ |\ \neg \phi\ |\ 
		\phi_1 \land \phi_2\ |\ \forall 
		x^{\sigma} . \phi\ |\ t_1 = t_2
		\ |\ P^{\sigma_1,...,\sigma_m}
		(t_1, ..., t_m)
	\end{align*}
	Terms are either sorted variables, 
	or applications of functions to terms.
	Formulas are constants or logical 
	connectives applied to other 
	formulas, quantified formulas, 
	equality over terms, or predicates 
	over terms. Connectives $\lor$, 
	$\to$, $\iff$, and the existential
	quantifier $\exists$ can be 
	specified using the connectives 
	and quantifiers mentioned above.
	
	\subsection{Translation}
	\label{sec:trans}
	Sledgehammer's SMT solver 
	integration (Bohme et al.) performs 
	a translation 
	of HOL formulas to MSFOL formulas.
	$\llbracket\ \rrbracket$
	is the translation function 
	that maps both HOL types and 
	terms to MSFOL sorts and terms,
	respectively.
	The translation is sound --- 
	given a formula 
	$F = \Gamma \land \Phi$ in HOL, if 
	$\llbracket F \rrbracket$ is refutable 
	($\neg \llbracket F \rrbracket$
	is unsatisfiable) in MSFOL, then 
	$F$	is valid in HOL --- but not 
	complete --- if $F$ is valid in 
	HOL, $\llbracket F \rrbracket$ is 
	not necessarily refutable in MSFOL. 
	The translation might not carry over 
	enough information about certain HOL 
	formulas so that the their 
	refutability can be concluded in 
	MSFOL, which is	expected, since HOL 
	is a much more expressive logic than 
	MSFOL. 
	
	Since MSFOL is a subset of 
	HOL, we can see HOL as a 
	composition of 
	MSFOL-equivalent logic 
	and the rest:
	\begin{center}
		HOL = MSFOL-equiv + non-MSFOL 
	\end{center}
	The MSFOL-equiv part of HOL has 
	a straightforward transformation 
	to MSFOL:
	\begin{itemize}
	\item Only nullary type 
		constructors from HOL have
		equivalent MSFOL sorts:
		\begin{center}
			$\llbracket \kappa^{0} 
			\rrbracket = \sigma $
		\end{center}
	\item Applications of 
		Boolean connectives,
		quantifiers, equalities and 
		other predicates (functions 
		returning \texttt{Bool}) to 
		terms have corresponding 
		MSFOL formulas:
		\begin{align*}
			\llbracket False 
			\rrbracket &\cong \bot \\
			\llbracket \neg t \rrbracket 
			&\cong \neg \llbracket t 
			\rrbracket\\
			\llbracket t \land u 
			\rrbracket &\cong \llbracket t 
			\rrbracket \land \llbracket u
			\rrbracket\\
			\llbracket t = u \rrbracket 
			&\cong \llbracket t 
			\rrbracket = \llbracket u
			\rrbracket\\
			\llbracket c^{\tau_1 \to ... 
			\to \tau_n \to \texttt{bool}} 
			t_1 ... t_n \rrbracket &\cong 
			c^{\llbracket \tau_1 \rrbracket, 
			..., \llbracket \tau_n \rrbracket}
			(\llbracket t_1 \rrbracket, ..., 
			\llbracket t_n \rrbracket)\\
			\llbracket \forall x^{\tau}.t 
			\rrbracket &\cong (\forall 
			x^{\llbracket \tau \rrbracket}.
			\llbracket t \rrbracket)
		\end{align*}
	\item Variables and function 
		applications (of return type 
		other than \texttt{bool}), 
		including non-functional 
		constants have MSFOL 
		equivalents.
		\begin{align*}
			\llbracket x^{\tau} 
			\rrbracket &\cong 
			x^{\llbracket \tau \rrbracket}\\
			\llbracket c^{\tau_1 \to ... 
			\to \tau_n \to \tau} 
			t_1 ... t_n \rrbracket &\cong 
			c^{(\llbracket \tau_1 \rrbracket, 
			..., \llbracket \tau_n \rrbracket)
			\to \llbracket \tau \rrbracket}
			(\llbracket t_1 \rrbracket, ..., 
			\llbracket t_n \rrbracket)
		\end{align*}
	\end{itemize}

	The more interesting parts of the 
	transformation deal with 
	translating the non-first-order
	elements of HOL to MSFOL. These 
	include:
	\begin{itemize}
		\item Type variables $\alpha$
		and \textit{compound types} ---
		applications of type constructors
		$\kappa^n$ with $n > 0$ to 
		types.
		\item $\lambda$-abstractions 
		such as $\lambda x. t$.
		\item Variables of functional 
		types and partial applications 
		of functions to arguments are 
		possible in HOL but not in 
		MSFOL. For example, 
		$t^{\tau_1 \to \tau_2 \to 
		\tau}\ t_1$ is a 
		partial application of $t$ 
		which is a variable of type 
		$\tau_1 \to \tau_2 \to \tau$
		to argument $t_1$ (of type 
		$\tau_1$) and this 
		application has type 
		$\tau_2 \to \tau$.
		This is not directly 
		expressible in MSFOL.
	\end{itemize}

	\subsubsection{Monomorphization}
		Recall that HOL types $\tau$ are 
		either type	variables $\alpha$ or 
		applications of type constructors 
		$\kappa^n$ to $n$ types.
		A \textit{monomorphic type} is a
		type without type variables 
		(e.g. $\kappa^0$, $\kappa^1 
		\kappa^0$, and $\texttt{bool}^0 
		\to \texttt{int}^0$ 
		where $\to$ is 
		a type consructor), and 
		a \textit{schematic type} is one 
		with type variables (such as 
		$\alpha \to \texttt{bool}^0$). 
		Monomorphization involves 
		repeatedly instantiating schematic
		terms based on a set of 
		monomorphic terms until a fixed 
		point is reached.
		
		The definition of monomorphization 
		requires a description of 
		\textit{instantiation} of schematic 
		entities w.r.t. 
		monomorphic ones.
		\begin{itemize}
		\item Informally, a monomorphic type 
			$\tau_M$ \textit{matches} a schematic 
			type $\tau_S$ if it can replace the 
			type variables in the schematic 
			type. For example, \texttt{bool} 
			matches $\alpha$; and $\texttt{bool} 
			\to \texttt{int}$ matches 
			$\texttt{bool} \to \beta$
			since \texttt{int} matches 
			$\beta$. Finally, for our running 
			example taken from Bohme et al., 
			${\texttt{bool} \to \kappa}$
			matches $\alpha \to \beta$,
			where $\kappa$ is a monomorphic
			type. 
			
		\item Given a monomorphic constant 
			$c^{\tau_M}$,  if $t^S$ is a 
			schematic term that contains a 
			schematic constant $c^{\tau_S}$ 
			such that $\tau_M$ matches 
			$\tau_S$, then $c^{\tau_M}$ induces 
			a substitution $\sigma$ on $t^S$, 
			and $\sigma(t^S)$ is an instance of 
			$t^S$ w.r.t. to $c^{\tau_M}$. 
			For example, the instance of term
			\begin{center}
				$(\forall f^{\alpha \to \beta},\ 
				x^{\alpha},\ xs^{\texttt{list }
				\alpha}.\ \texttt{apphd }f\ 
				(\texttt{cons }x\ xs) = f\ x)$
			\end{center} 
			w.r.t. constant
			\begin{center}
				$(\lambda x^{\texttt{bool}}.
				\texttt{if }x\texttt{ then }
				a^{\kappa} \texttt{ else }
				b^{\kappa})^{\texttt{bool} \to 
				\kappa}$ 
			\end{center}
			is
			\begin{center}
				$(\forall f^{\texttt{bool} 
				\to \kappa},\ x^{\alpha},\ 
				xs^{\texttt{list }\alpha}. 
				\texttt{ apphd }f\ (\texttt{cons }
				x \ xs) = f\ x)$
			\end{center}
			where the instantiation 
			has turned all occurrences of $f$ 
			in the term from $f^{\alpha \to \beta}$ 
			to $f^{\texttt{bool} \to \kappa}$.
			The following presents some inline, 
			technical preliminaries about 
			the rest of the terms.
			\texttt{list $\alpha$} is a compound,
			polymorphic type of a list of 
			$\alpha$'s where $\alpha$ is a type 
			variable. \texttt{list bool} is the 
			type of lists of Booleans. \texttt{cons} 
			is the list constructor of type 
			$\alpha \to \texttt{list }\alpha \to 
			\texttt{list }\alpha$ and $[\ ]$ is 
			the list constructor for empty lists. 
			For example, the integer list 
			containing $1$, $2$, and $3$ (in that 
			order) is represented as 
			$\texttt{cons }1\ (\texttt{cons }2
			\ (\texttt{cons }3\ [\ ]))$.
			\texttt{hd} is a $\texttt{list}\ 
			\alpha \to \alpha$ function that 
			returns the first element of a list 
			and \texttt{apphd} is an $(\alpha
			\to \beta) \to \texttt{list}\
			\alpha \to \beta$ function that
			takes a function $f$ and a list 
			$l$ as input, and applies $f$
			to the head of $l$ (or, $f\ 
			(\texttt{hd }l)$).
		\item The idea of an instance is 
			extended to terms as follows. If 
			$t^M$ is a monomorphic term, then 
			$\sigma(t^S)$ is an instance of 
			$t^S$ w.r.t. $t^M$, if 
			$\sigma$, the combination of all 
			substitutions induced by the 
			constants in $t^M$ is defined. This 
			instance might still be schematic. For 
			example, since $x^{\texttt{bool}}$ is 
			an instance of $x^{\alpha}$ w.r.t.
			$\texttt{T}^{\texttt{bool}}$, 
			and $xs^{\texttt{list bool}}$ is 
			an instance of 
			$xs^{\texttt{list }\alpha}$ 
			w.r.t. $[\ ]^{\texttt{list bool}}$,
			we can combine the instantiation 
			of the constant in the previous 
			example to obtain that the 
			instantiation of term
			\begin{center}
				$(\forall f^{\alpha \to \beta},\ 
				x^{\alpha},\ xs^{\texttt{list }
				\alpha}.\ \texttt{apphd }f\ 
				(\texttt{cons }x\ xs) = f\ x)$
			\end{center}
			w.r.t. the term
			\begin{center}
				$(\texttt{apphd }(\lambda 
				x.\ \texttt{if }x\texttt{ then }
				a \texttt{ else } b)^{\texttt{bool} 
				\to \kappa}\ (\texttt{cons T}\ 
				[\ ])) \neq a$
			\end{center}
			to be
			\begin{center}
				$(\forall f^{\texttt{bool}
				\to \kappa},\ x^{\texttt{bool}},
				\ xs^{\texttt{list bool}}.\ 
				\texttt{apphd }f\ (\texttt{cons }x
				\ xs) = f\ x)$
			\end{center}
		\item Instantiation can further be 
			extended to sets of schematic 
			terms $S$ and monomorphic terms $M$. 
			An instance of $S$ w.r.t. 
			$M$ is the set $I$ of terms such 
			that each term in $I$ is an instance 
			of some term from $S$ w.r.t. 
			some term from $M$. Since 
			instantiation can produce either 
			monomorphic or schematic terms, $I$
			can be partitioned into either as
			$(I_M, I_S)$.
		\end{itemize}
		Now, we can describe monomorphization.
		\begin{itemize}
		\item A \textit{monomorphization 
			step} for $S$ w.r.t. $M$ maps 
			the pair $(M,S)$ to the pair 
			$(M \cup I_M, S \cup I_S)$.
		\item The complete 
			\textit{monomorphization} of $S$ 
			w.r.t. $M$ is the computation of 
			a least fixed point of 
			monomorphization steps of $S$ 
			w.r.t. $M$.
		\item Given a HOL formula $F$, 
			if $(M, S)$ is the partition of its 
			constituents into monomorphic and 
			schematic terms, then 
			monomorphization of $S$ 
			w.r.t. $M$ yields pair 
			$(M^{\prime}, S^{\prime})$ and we 
			call the conjunction of all terms in 
			$M^{\prime}$ the monomorphization 
			of $F$.
		\end{itemize}
		Thus, using the above examples as 
		instantiation steps, we have that 
		monomorphization translates the formula $F$
		\begin{align*}
			F:\ &(\forall f^{\alpha \to \beta},\ 
			x^{\alpha},\ xs^{\texttt{list }\alpha}.\ 
			\texttt{apphd }f\ (\texttt{cons }x
			\ xs) = f\ x)\ \land\ \\
			&((\texttt{apphd }(\lambda x.\ 
			\texttt{if }x \texttt{ then }a 
			\texttt{ else } b)^{\texttt{bool} 
			\to \kappa}\ (\texttt{cons T}\ [\ ])) 
			\neq a)
		\end{align*}
		to the formula $F^{\prime}$
		\begin{align*}
			F^{\prime}:\ &(\forall 
			f^{\texttt{bool} \to \kappa},\ 
			x^{\texttt{bool}},\ 
			xs^{\texttt{list bool}}.\ 
			\texttt{apphd }f\ (\texttt{cons }x
			\ xs) = f\ x) \ \land\ \\
			&((\texttt{apphd } (\lambda x.\ 
			\texttt{if }x \texttt{ then }a 
			\texttt{ else } b)^{\texttt{bool} 
			\to \kappa}\ (\texttt{cons T}\ 
			[\ ])) \neq a).
		\end{align*}
		\noindent The following are some 
		limitations of this process:
		\begin{itemize}
		\item There have to be monomorphized 
			terms in a formula, to guide the 
			monomorphization process, so it 
			seems like this process would 
			fail with formulas that contain 
			only schematic terms. 
		\item The monomorphization process 
			could be non-terminating. In 
			other words, the first component
			of the pair yielded by 
			monomorphization --- the 
			set of monomorphic terms ---
			could be infinite. For example,
			consider $S = \{c^{\alpha}
			\land c^{\kappa\ \alpha}\}$
			and $M = \{c^{\kappa_0}\}$.
			Now, $\kappa_0$ matches 
			$\alpha$, so the instance
			$c^{\kappa_0} \land 
			c^{\kappa\ \kappa_0}$ is added 
			to $M$ after a monomorphization 
			step. Now, $\kappa\ \kappa_0$
			matches $\alpha$, so the 
			instance $c^{\kappa\ \kappa_0} 
			\land c^{\kappa\ \kappa\ 
			\kappa_0}$ is added to $M$, 
			and so on. Since the proof of a 
			formula, if it existed, would 
			be finite, most monomorphic 
			terms would be irrelevant. 
			Finding the finite subset of 
			necessary monomorphic terms is 
			undecidable~\cite{10.1007/978-3-642-24364-6_7},
			but Bohme et al. use heuristic
			methods to overapproximate
			this set. They limit the 
			number of monomorphization 
			steps and the number of 
			monomorphic terms generated
			with the expectation that 
			monomorphic terms that 
			contribute to proofs 
			are typically generated early 
			in the monomorphization process.
		\item The monomorphization process
			is described as a syntactic 
			process. Semantic steps in 
			the process aren't described, 
			and if it doesn't indeed 
			involve	any semantic pruning of 
			translations, the number of 
			monomorphization steps 
			might be impractically large. 
			For example, from the formula 
			in the running example above ($F$), 
			$x^{\alpha}$ could match 
			$a^{\kappa}$ and be instantiated 
			to $x^{\kappa}$. While 
			syntactically, this would 
			check out, semantically, 
			given that $f^{\alpha \to \beta}$
			is instantiated to 
			$f^{\texttt{bool} \to \kappa}$, 
			and that $f$ is applied to $x$, 
			$x$ has to have type 
			\texttt{bool} and this can be 
			achieved by matching it with 
			$x^{\texttt{bool}}$.
		\item Monomorphization is 
			incomplete and the instantiations 
			are done heuristically. This 
			means that if the SMT solver 
			is not able to prove the 
			monomorphized version of a 
			formula, a different 
			monomorphization could 
			very well be provable. Thus, 
			the monomorphization of a 
			problem would have to be 
			done smartly, and sometimes
			repeated multiple times to 
			be successful. 
		\end{itemize} 
	
	\subsubsection{Lambda-Lifting}
		$\lambda$-abstractions represent 
		anonymous or unnamed functions, 
		which aren't allowed in MSFOL.
		These abstractions are removed
		from HOL formulas by a process 
		called \textit{$\lambda$-lifting}
		which uses a fresh constant as a
		name for the abstraction and adds 
		a quantified formula specifying 
		it's behavior. Concretely,
		\begin{align*}
			\llbracket t[\lambda x^{\tau}.u]
			\rrbracket \cong 
			(t[(\lambda x^{\tau}.u) \mapsto c]
			\land (\forall x^{\tau}.\ c\ x = u))
		\end{align*}
		The notation $t[x]$ represents a 
		term $t$ with a sub-term $x$, 
		and $t[x \mapsto y]$ is the 
		term obtained by substituting all 
		occurrences of $x$ by $y$ in $t$.
		$c$ is specified in MSFOL as an 
		uninterpreted function with sort 
		$\tau \to \kappa$ where $\kappa$ 
		is the sort of $u$. For instance, 
		$(\lambda x^{\texttt{int}}, x + 1)\ 
		5 = 6$
		is translated to $(c\ 5 = 6) \land
		(\forall x^{\texttt{int}}.\ 
		c\ x = x + 1)$. $\lambda-$lifting
		faithfully preserves the 
		semantics of the function, and 
		thus it maintains soundness. 
		Quantified formulas are usually a 
		source of undecidability for SMT 
		solvers, so these lifted formulas 
		contribute to the incompleteness
		of the translation.
		%If the 
		%$\lambda-$ abstraction contains free
		%variables, then they are turned 
		%into additional arguments to $c$.
		
	\subsubsection{Explicit Applications}
		For functional constants 
		occurring multiple times with 
		a variable number 
		of arguments, the minimal 
		number of arguments are 
		considered as the arity of the 
		function, and any additional 
		arguments are expressed 
		explicitly, with the help of a 
		constructor 
		$\texttt{app}^{(\alpha_1 \to 
		\alpha_2) \to \alpha_1 \to 
		\alpha_2}$ that is defined as:
		\begin{center}
			$\texttt{app }t_1\ t_2 = 
			t_1\ t_2$
		\end{center}
		If $c$ is a constant that 
		occurs with at least $n$
		arguments, then 
		occurrences of applications 
		of $c$ are translated as:
		\begin{center}
			$\llbracket c\ t_1\ ...\ t_n
			\ u_1\ ...\ u_m \rrbracket
			\cong \texttt{app}\ (...\ 
			(\texttt{app }(c\ t_1\ 
			...\ t_n)\ u_1)	\ ...)\ u_m$
		\end{center}
		where $m$ could be $0$, in which 
		case there are no applications 
		of \texttt{app}. For example, if 
		$f^{\texttt{int} \to \texttt{int}}$ 
		occurs twice in a formula, once 
		partially applied to no arguments 
		as $f$, and once fully applied as 
		$(f\ 0)$, then since the minimal 
		number of arguments to $f$ is 
		zero, we have:
		\begin{align*}
		\llbracket f \rrbracket &\cong
		f\\
		\llbracket f\ 0 \rrbracket &\cong
		\texttt{app } f\ 0
		\end{align*}
		
		\texttt{app} is a higher-order 
		function so both the function 
		($f$ in the above example) and 
		\texttt{app} have to be encoded 
		in the SMT solver as constructors.
		This can also be done using arrays 
		which are essentially functional 
		types which can be passed
		around, since arrays in 
		SMTLIB~\cite{BarFT-SMTLIB} (the 
		standard for SMT solvers)
		can have any index and value types, 
		and map elements of the index type 
		to the value type. Bohme et al. 
		don't specify what technique 
		they use to encode the 
		\texttt{app} constructor.
		
		Using the \texttt{app} constructor
		for partially applied interpreted 
		constants in MSFOL, such as 
		logical connectives, would result 
		in terms that aren't well-typed.
		These are $\eta-$expanded and 
		then $\lambda-$lifted. For example, 
		for the partially applied 
		conjunction ($\land$, used as prefix 
		here), a partial application is 
		translated as follows.
		\begin{align*}
			\llbracket t [\land\ x] 
			\rrbracket &\cong \llbracket 
			 t[\lambda y.\ \land\ x\ y]
			\rrbracket&(\eta-expansion)\\
			&\cong t[c]  \land (\forall y.\ 
			c\ y = \land\ x\ y) &(\lambda-lifting)
		\end{align*}
		
	\subsubsection{Erasure of Compound Types}
		A type constructor $\kappa^n$ with 
		$n > 0$ is applied to $n$ types 
		$\tau_i$. After monomorphization, 
		each $\tau_i$ is monomorphic, so 
		$\kappa^n\ \tau_1\ ...\ \tau_n$ 
		is a monomorphic, compound type. It
		is represented as a fresh nullary 
		type constructor $\kappa_0^n$ with 
		the same interpretation as 
		$\kappa^n\ \tau_1\ ...\ \tau_n$, 
		which can now be represented in 
		MSFOL.
		\begin{center}
			$\llbracket \kappa^n\ 
			\tau_1\ ...\ \tau_n \rrbracket
			\cong \kappa_0^n$
		\end{center}
		Some examples are:
		\begin{align*}
			\llbracket \texttt{bool} \to
			\kappa \rrbracket &\cong \kappa_1\\
			\llbracket \texttt{list\ bool}
			\rrbracket &\cong \kappa_2
		\end{align*}
		At the time of writing of Bohme et al.,
		SMT solvers didn't support sorts 
		parameterized by other sorts, but they 
		support them now, and these can be 
		used to encode compound types.
		
		While this is a straightforward
		translation step syntactically, 
		notice that SMT solvers need 
		extra help via premise selection 
		to solve constraints involving 
		these types. Integers are 
		axiomatized in the SMT solver as 
		the theories of linear and 
		non-linear integer arithmetic, and 
		Isabelle/HOL integers are made to 
		correspond to these in the translation, 
		so when a formula containing 
		integers is sent to the SMT solver, 
		it can perform integer reasoning. 
		However, SMT solvers have no 
		axiomatization of, say, lists 
		of Booleans. It is crucial that when 
		\texttt{list bool} is 
		translated to $\kappa_2$, the 
		premise selector of sledgehammer
		selects the right lemmas about 
		Boolean lists to give to the 
		SMT solver so it can reason 
		about them. So in this step, 
		completeness depends heavily on 
		premise selection.
		
	\subsubsection{Interpreted and Uninterpreted Constants}
		As mentioned in the previous section,
		certain types in Isabelle/HOL correspond
		to certian sorts in the SMTLIB standard.
		Bohme et al. exploit the axiomatizations 
		of these sorts by translating constants 
		of these types from HOL to the 
		corresponding MSFOL constants - these 
		include linear arithmetic constants over 
		integers and reals and bit-vector 
		arithmetic. They even use the SMTLIB 
		theory of algebraic 
		datatypes~\cite{BarST-PDPAR-06} to 
		encode inductive types from Isabelle/HOL
		such as lists. This allows the SMT solver
		to rely on the decision procedures for 
		these theories and not just the premises
		provided by Sledgehammer, to prove the 
		conjecture. 
		
		The rest of the constants are 
		translated as uninterpreted constants, 
		and this is a source of 
		incompleteness --- the SMT solver only 
		knows as much about an uninterpreted 
		constants as the additional constraints 
		(premises) will tell it. If the premise 
		selector of Sledgehammer doesn't capture 
		the constraints needed to prove something
		about an uninterpreted constant, the SMT 
		solver will not be able to prove it.
		Even with interpreted constants, if the 
		constants belong to an undecidable theory,
		the SMT solver will not necessarily be 
		able to prove it.
		
\bibliographystyle{abbrv}
\bibliography{bib2}

\end{document}