\documentclass{article}
\usepackage{proof}
\usepackage{bussproofs}
\usepackage{xcolor}
\usepackage{url}
\usepackage{framed}
\usepackage{amsmath, amsfonts, amssymb}
\usepackage{mathtools}
\usepackage{minted}
\usepackage{stmaryrd}
\usepackage{verbatim}

\usepackage{hyperref}
\hypersetup{
 pdfborder={0 0 0},
 colorlinks=true,
 linkcolor=blue,
 urlcolor=blue,
 citecolor=blue
}
\usepackage{cleveref}


\newcommand{\rem}[1]{\textcolor{red}{[#1]}}
\newcommand{\ed}[1]{\textcolor{blue}{#1}}
\newcommand{\ct}[1]{\rem{#1 --ct}}

\begin{document}
\title{Comprehensive Exam Report 3 - Proof Generation}
\author{Arjun Viswanathan}
\date{}
\maketitle

\section{Introduction}
\label{sec:intro}
	SMTCoq~\cite{DBLP:phd/hal/Keller13} 
	is a skeptical cooperation 
	between the Coq proof assistant, and 
	(SAT and) SMT solvers~\cite{Barrett2018}, 
	implemented as a 
	Coq plugin. This integration is 
	useful for the ITP (Coq) since it 
	allows proof goals in Coq to be 
	automatically discharged by an 
	SMT solver. However, unlike 
	Coq's small, verifiable kernel,
	SMT solvers have tens of thousands 
	of lines of code and are more prone
	to bugs. Since an SMT solver's code 
	base is much harder to verify, 
	many SMT solvers produce proofs
	of their results in addition to 
	a claim of (un)satisfiability
	of a set of formulas. For satisfiable 
	formulas, a proof is simply a
	satisfying model which is easy 
	to specify and check (for 
	quantifier-free formulas). For 
	unsatisfiable formulas, however,
	SMT solvers need to show that 
	the input formulas are inconsistent.
	They do this by reducing the formulas 
	to a trivial unsatisfiable formula
	$False$, or $\bot$. A proof 
	producing SMT solver is integrated 
	with Coq by checking the proof 
	inside Coq for a particular 
	representation of SMT formula in 
	Coq. The checker is proven correct,
	lifting up the formulas
	to Coq formulas via a
	process called computational 
	reflection (described in Report 2).
	
	An added complication in this 
	integration of SMT solvers with 
	Coq is the incompatibility of 
	their logics. Propositions in 
	Coq have the 
	\texttt{Prop} type which are 
	implemented a constructive
	or intuitionistic logic. SMT
	solvers, on the other hand, 
	implement a classical logic. This
	incompatibility is overcome in 
	SMTCoq by using SMT solvers to 
	prove Boolean formulas in 
	Coq (of type \texttt{Bool})
	that are classical, while also 
	supporting a fragment of 
	propositions that don't violate 
	the laws of constructive logic.
	
	The first half of this report 
	discusses the proof production 
	systems of SMT solvers. The
	second half explores the difference
	between the logics of SMT solvers
	and Coq, and SMTCoq's efforts 
	to prove formulas in Coq that 
	SMT solvers can help with,
	while being logically consistent.

\section{Preliminaries}
\label{sec:prelims}
	Many-sorted first-order logic extends
	first-order logic (FOL) with 
	types or sorts. We present the 
	syntax in this section and the 
	semantics are presented in
	\cite{Barrett2018}. Syntactically, 
	the components of MSFOL are sorts 
	$\sigma$, terms $t$, and 
	formulas $\phi$. Sorts are 
	atomic entities that 
	represent types. Function types 
	--- ($\sigma_1$, ..., $\sigma_n$) 
	$\to$ $\sigma$ ---
	and relation types 
	--- ($\sigma_1$, ..., $\sigma_m$)
	are defined over sorts, and 
	are types of functions and 
	predicates below. Terms $t$ and 
	formulas $\phi$ are specified as:
	\begin{align*}
		t &:= x^{\sigma}\ |\ 
		f^{(\sigma_1, ..., \sigma_n) \to 
			\sigma}	(t_1, ..., t_n)\\
		\phi &:= \bot\ |\ \neg \phi\ |\ 
		\phi_1 \land \phi_2\ |\ \forall 
		x^{\sigma} . \phi\ |\ t_1 = t_2
		\ |\ P^{\sigma_1,...,\sigma_m}
		(t_1, ..., t_m)
	\end{align*}
	Terms are either sorted variables, 
	or applications of functions to terms.
	Formulas are constants or logical 
	connectives applied to other 
	formulas, quantified formulas, 
	equality over terms, or predicates 
	over terms. Connectives $\top$, $\lor$, 
	$\Rightarrow$, $\leftrightarrow$, and the 
	existential	quantifier $\exists$ 
	can be specified using the connectives 
	and quantifiers mentioned above.
	
	A \textit{literal} is a term, or a 
	formula with no connectives other than 
	$\neg$. A \textit{clause} is a 
	disjunction of literals. All formulas 
	can be converted to conjunction normal 
	form (CNF) which is a conjunction of 
	clauses. For ease of notation, 
	a formula in CNF is also represented
	as a set of clauses within curly 
	braces. $\bot$, the formula 
	representing Boolean false, 
	is overloaded to also represent
	the empty clause, which is trivially
	unsatisfiable (unlike the empty set,
	or the empty conjunction which is 
	trivially satisfiable).
	
	Informally, a formula is satisfiable
	if there exists a satisfying model 
	--- an assignment of well-sorted 
	values to its variables such that 
	the interpretation of the formula 
	is $\top$; it is unsatisfiable if 
	no such model exists; it is valid
	if for all models its interpretation
	is $\top$. This interpretation 
	is recursively defined in 
	Barrett et al.~\cite{Barrett2018}.
	We use the notation 
	$var \to val$ to signify the 
	assigment of value $val$ to 
	variable $var$, and group 
	assignments in curly braces. 
	For example, 
	$a \to True$ assigns Boolean value 
	$True$ to $a$. $\models$ stands for 
	logical entailment and $\models_T$ 
	stands for entailment in theory $T$. 
	A formula is entailed from nothing, 
	if it is valid. For example, 
	$a \land b \models a$,
	$x > 3 \models_{LIA} x > 0$, and 
	$\models_{LIA} x > 3 \Rightarrow 
	x > 0$ are all logically sound 
	entailments --- the first one 
	propositionally, and the
	last two in the theory of linear 
	integer arithmetic.
	
	Declarative rules of the form
	\begin{center}
		$\infer[rule\_name]{C}{P_1 & ... & P_n}$
	\end{center}
	specify that if the premises $P_1$ to 
	$P_n$ hold, then the conclusion $C$
	holds.
	
	

\section{DPLL(T) Proofs}
\label{sec:smtproofs}
	The typical SMT solver composes 
	a SAT solver with multiple theory
	solvers in an abstraction-refinement
	cycle, where the SAT engine 
	tries to find a satisfying 
	assignment of the propositional
	abstraction of a given set of 
	constraints, and the theory
	solvers find a refutation of 
	the refinement, if one exists.
	This cycle is guided by the 
	DPLL(T) algorithm, which is an
	extension of the 
	DPLL~\cite{Loveland2016}
	(Davis-Putnam-Logemann-Loveland) 
	algorithm that adds theory-level 
	reasoning. DPLL(T) is a 
	resolution-based calculus and the 
	resolution rule is central to its 
	reasoning:
	\begin{center}
		$\infer[resolution]{\phi_1 \lor ... \lor 
			\phi_n \lor \psi_1 \lor ... \lor 
			\psi_m}{\phi_1 \lor ... \lor \phi_n 
			\lor \chi & \neg \chi \lor \psi_1 
			\lor ... \lor \psi_m}$ 
	\end{center}
	The following is an instance of the 
	resolution rule.
	\begin{center}
		$\infer{a \lor c}{a \lor \neg b 
			& b \lor c}$
	\end{center}
	Given that two clauses (a clause 
	is a disjunction of literals) hold, 
	and a \textit{pivot}, which is a 
	literal that occurs with opposite 
	polarities in each clause ($b$ in 
	the above example), the resolution 
	rule concludes that 
	the combination of the two clauses 
	without either occurrence of the 
	pivot holds.
	
	An SMT solver converts its input 
	constraints into conjunction 
	normal form (CNF), which is a 
	conjunction of clauses. The 
	DPLL(T) algorithm then tries 
	to find a satisfying assignment 
	for these clauses, and in the 
	case that one doesn't exist,
	resolves these clauses to 
	conclude the empty clause 
	$\bot$ from them, which is the 
	most basic form of unsatisfiability.
	
	A proof of unsatisfiability is 
	a tree with the input clauses at 
	its leaves, and $\bot$ at the 
	root, where clauses are reduced
	via the resolution rule. In 
	addition to the input clauses, 
	leaves consist of clauses 
	called \textit{theory lemmas}, 
	which are the theory solvers' 
	means of refining the SAT solver's
	model (a model assigns values 
	to variables) for the input. A
	theory lemma, produced by the 
	corresponding theory
	solver, is a clause that is
	valid in a theory. The theory 
	solvers use theory-specific 
	rules in addition to the 
	resolution rule to prove these 
	lemmas. A theory lemma in the 
	proof tree is supported by a 
	satellite proof of the lemma
	provided by the theory solver. 
	
	In what follows, we explain the 
	DPLL(T) algorithm and the 
	extraction of a resolution proof
	from an execution of it on an 
	unsatisfiable input.
	
	\subsection{The DPLL(T) Algorithm}
	\label{sec:dpllt}
	
	DPLL(T) is originally presented 
	in Nieuwenhuis et al.~\cite{DBLP:journals/jacm/NieuwenhuisOT06}, 
	and reiterated in Katz et 
	al.~\cite{DBLP:conf/fmcad/KatzBTRH16} 
	as a transition system. We briefly present 
	the transition system here. Katz et al. also 
	present the proof-production system of the 
	CVC4~\cite{DBLP:conf/cav/BarrettCDHJKRT11} 
	SMT solver that we follow 
	in this report. While other SMT solvers'
	proofs might vary in specifics from 
	CVC4's, the general idea of a 
	resolution proof is still common 
	across solvers.
	
	As previously mentioned, the DPLL(T)
	algorithm can be understood as 
	trying to satisfy the input constraints
	at the propositional level, with 
	refinements from theory solvers
	about models. Checking whether a
	formula is unsatisfiable at the 
	propositional level involves 
	incrementally assigning literals
	deterministically to satisfy clauses
	(propagation), and when that isn't an 
	option, by guessing an 
	assignment. When a guess goes 
	down a path that results in 
	unsatisfiability, the algorithm 
	backtracks and checks whether the 
	formula is still unsatisfiable 
	when the guess is flipped.
	
	We now formalize the DPLL(T) algorithm 
	as a state transition systems that 
	operates on a set of transition rules. 
	The state of the system is either 
	$fail$ or $\langle M, F, C \rangle$ 
	where $M$ is the current 
	\textit{context}, that is, 
	the current assignment of literals in 
	the formula; a literal in $M$ is 
	preceded by a $\bullet$ if the literal 
	was a decision, that is, it was 
	guessed. For some variable $x$, 
	the assignment $x \to True$ is added 
	to the context by appending $x$ to 
	$M$ and $x \to False$ is added by 
	appending $\neg x$. $F$ is a set of 
	clauses representing the input formula 
	in CNF. SMT solvers generally convert 
	arbitrary inputs to CNF and thus 
	their proofs also involve this CNF
	conversion phase, but we omit it here.
	$C$ represents the conflict 
	clause, the clause from $F$ that is 
	falsified by the current assignment. 
	$\prec_M$ is an ordering 
	on the literals in $M$ ---
	$l_1, ..., l_n \prec_M l$ if 
	$l_1, ..., l_n$ occur before 
	$l$ in $M$. If $M = M_0 
	\bullet M_1 \bullet ... \bullet M_n$, 
	such that no $\bullet$ occurs in 
	the $M_i$'s, then each $M_i$ is a
	\textit{decision level}, $M^{[i]}$ 
	denotes $M_0 \bullet ... \bullet M_i$,
	and and \texttt{lev} is a function 
	that maps a literal in $M$ to its 
	decision level. $Lit_F$ represents the 
	set of literals occurring in $F$.
	
	The initial state of the system is 
	$\langle \phi, F, \phi \rangle$. 
	The final state is one of:
	\begin{itemize}
		\item $fail$, when $F$ is 
		unsatisfiable.
		\item $\langle M, F_0, \Phi \rangle$ 
		where $F$ is satisfiable, $F_0$ is 
		a transformation of $F$ by the 
		algorithm that maintains its 
		(un)satisfiability,	and $M 
		\models F_0$.
	\end{itemize}
	
	The transition system goes from the 
	initial state to one of the possible 
	final states by a nondeterministic 
	application of the transition rules 
	presented in Figure~\ref{fig:proprules} 
	and Figure~\ref{fig:theoryrules}. In 
	practice, they are applied using a 
	particular strategy for efficiency.
	A rule is applicable when all its 
	premises hold, and it is applied to 
	make the conclusion true. 

	\begin{figure}[t]
	\centering
	\begin{framed}
	\begin{tabular}{c}
		$\infer[Prop]{M := Ml}{l_1 \lor ... \lor 
			l_n \lor l \in F & \neg l_1,...,\neg 
			l_n \in M & l, \neg l \notin M}$ \\ \\
		$\infer[Dec]{M := M \bullet l}{l \in 
			Lit_F & l, \neg l \notin 
			M}$ \\ \\
		$\infer[Confl]{C:=\{l_1 \lor ... \lor 
			l_n\}}{C = \phi & l_1 \lor ... \lor 
			l_n \in F & \neg l_1,...,\neg l_n \in 
			M}$ \\ \\
		$\infer[Expl]{C := \{l_1 \lor ... \lor 
			l_n \lor D\}}{C = \{\neg l \lor D\} & 
			l_1 \lor ... \lor l_n \lor l \in F & 
			\neg l_1,...,\neg l_n \prec_M l}$ \\ \\
		$\infer[Backj]{C := \phi\ \ \ M := M^{[i]}l}
		{C = \{l_1 \lor ... \lor l_n \lor l\} & 
			\texttt{lev}\ \neg l_1, ... , 
			\texttt{lev}\ \neg l_n \leq 
			i < \texttt{lev}\ \neg l}$ \\ \\
		$\infer[Learn]{F := F \cup C}{C \neq 
			\phi}$ \\ \\
		$\infer[Fail]{fail}{C \neq \phi & \bullet 
			\notin M}$ \\ \\
	\end{tabular}
	\end{framed}
	\caption{Transition rules at Propositional level}
	\label{fig:proprules}
	\end{figure}

	\begin{figure}[t]
	\centering
	\begin{framed}
	\begin{tabular}{c}
		$\infer[Prop_i]{M := Ml}{l \in Lit_F 
			 & \models_i l_1 \lor 
			... \lor l_n \lor l & \neg l_1, 
			..., \neg l_n \in M & l,\neg l 
			\notin M}$ \\ \\
		$\infer[Confl_i]{C := \{l_1 \lor ... 
			\lor l_n \}}{C = \phi & \models_i 
			l_1 \lor ... \lor l_n & \neg l_1, 
			..., \neg l_n \in M}$ \\ \\
		$\infer[Expl_i]{C := \{l_1 \lor ... 
			\lor l_n \lor D\}}{C = \{\neg l 
			\lor D\} & \models_i l_1 \lor ... 
			\lor l_n \lor l & \neg\ l_1, ..., 
			\neg\ l_n \prec_M l}$ \\ \\
	\end{tabular}
	\end{framed}
	\caption{Transition rules at Theory level}
		\label{fig:theoryrules}
	\end{figure}

	The propositional rules from
	Figure~\ref{fig:proprules}
	model the behavior of the SAT solver.
	Propagations ($Prop$) force 
	assignments of Boolean values to literals 
	that are propositionally entailed. 
	For instance, if the input 
	formula is $\neg a \land (a \lor b)$, 
	then the assignment $a \to False$ is 
	required to satisfy the first conjunct,
	$\neg a$. Now, $b \to True$ is the only 
	way to satisfy the second conjunct, 
	$(a \lor b)$. Both of these assignments 
	are logically entailed propagations, 
	given the goal of satisfying the formula. 
	Sometimes a deterministic choice 
	might not be entailed, instead, 
	an assignment must be guessed 
	using the decide rule ($Dec$). For 
	example, to 
	satisfy $(\neg a \lor b) \land 
	(\neg b \lor c) \land (\neg c \lor 
	\neg a)$, a formula 
	with no possible propagations, 
	one of the literals must be 
	non-deterministically assigned.
	For instance, the guess $a \to True$
	forces propagations $b \to True$ and 
	$c \to False$. Propogations and/or 
	decisions could lead to a model in 
	which the input formula is not satisfied. 
	The conflict rule ($Confl$) recognizes 
	this. In the previous example, the 
	decision $a \to True$ led via 
	propagations to the assignment
	$\{a \to True,\ b \to True,\ 
	c \to False\}$, which falsifies 
	the second conjunct of the formula 
	and satisfies the others. The 
	$Confl$ rule recognizes this 
	clause ($\neg b \lor c$) to be the 
	conflict clause. Once a conflict is 
	encountered, where no previous 
	decisions were made, then the 
	conflict was entailed, and the 
	unsatisfiability of the input formula
	can be concluded via the $Fail$ rule. 
	If, there are previous decisions at
	the time of conflict, then the backjump
	rule ($Backj$) picks a decision that is 
	most likely the one that resulted in 
	the conflict, rewinds the state to 
	that decision and flips the decision. 
	To do this, it might need the explain 
	rule ($Expl$) to modify the 
	conflict clause using resolution. 
	Since the conflict clause is initially 
	a clause in the input, and it is only 
	modified by resolving it with other 
	input clauses, it is always entailed 
	by the input. At any point, a modified 
	conflict clause may be learned 
	($Learn$) by adding it to the 
	set of clauses in the input formula. 
	The restriction in the $Backj$ rule 
	that requires a literal in the 
	conflict clause to be at least 
	one decision level above the rest of 
	the clause, forces the transition 
	system to apply the $Expl$ rule 
	so that the algorithm is able to 
	backtrack further than chronological
	backtracking (where the most 
	recent decision is flipped), to 
	reach the source of the conflict.
	In our example, since there is only 
	one decision so far, the algorithm
	backjumps to the assignment of 
	$a$ and flips it to $a \to False$. The 
	propagations $b \to True$ and $c \to 
	False$ resulted from the assignment 
	$a \to True$, and these are undone
	by the backjump. After flipping our 
	initial decision, we have $a \to 
	False$ which satisfies
	the first and third clauses of the 
	formula $(\neg a \lor b) \land 
	(\neg b \lor c) \land (\neg c 
	\lor \neg a)$. Since there 
	are no further propagations, the 
	decision $b \to False$ can be made 
	to satisfy the second clause, and the 
	entire formula. The theory rules
	from Figure~\ref{fig:theoryrules} model 
	the behavior of an abstract 
	theory solver. The $Prop_i$, 
	$Confl_i$, and $Expl_i$ rules
	work exactly like their 
	propositional variants, except 
	that they can use theory lemmas, 
	that are valid clauses in a 
	theory produced by the theory
	solver. 
	
	The transition system is 
	\textit{refutationally sound}:
	if an execution starting with 
	$\langle \phi, F, \phi \rangle$ ends 
	with $fail$, then $F$ is 
	unsatisfiable in $T$.
	
	\subsection{DPLL(T) Transition Example}
	\label{sec:transex}
	We now look at an execution of the 
	DPLL(T) algorithm, as presented in the 
	previous section, on an example input
	that demonstrates the usage of 
	the rules. Consider the following 
	CNF formula:
	\begin{center}
		$ F: \{f(g(z)) = h(y) \lor x \neq y,\ \ 
		f(g(z)) \neq h(y) \lor x \neq y,\ \ 
		x = y \lor f(m) = f(n),\ \ 
		f(m) = x,\ \ f(n) = y\}$
	\end{center}
	Each term contains 
	additional information in the theory 
	of equality over uninterpreted 
	functions (EUF) where the only 
	predicate is equality, and all
	functions are uninterpreted. The 
	DPLL(T) algorithm
	works by abstracting away all the 
	theory functionality into 
	propositions, and then 
	reasoning propositionally in 
	concert with theory solvers. 
	Consider the following propositional
	abstraction of $F$.
	\begin{center}
		$\ F: \{A \lor \neg B,\ \neg A \lor 
			\neg B,\ B \lor C,\ D,\ E\}$
	\end{center}

	An execution of the DPLL(T) algorithm
	from the previous section on $F$, from 
	initial to final state is presented 
	in Figure~\ref{fig:dplltex}.
	\begin{figure}[t]
	\begin{center}
	\begin{tabular}{l l l l l}
		\textbf{M} & \textbf{F} & \textbf{C} 
		& \textbf{Rule} & \textbf{Step}\\
		\hline
		& $A \lor \neg B, \neg A \lor 
			\neg B, B \lor C, D, E$ 
			& $\phi$ & Prop $(D)$ & 1\\
		$D$ & $A \lor \neg B, \neg A \lor 
			\neg B, B \lor C, D, E$ & 
			$\phi$ & Prop $(E)$ & 2\\
		$DE$ & $A \lor \neg B, \neg A \lor 
			\neg B, B \lor C, D, E$ & 
			$\phi$ & Dec $(A)$ & 3\\
		$DE\bullet A$ & $A \lor \neg B, 
			\neg A \lor \neg B, B \lor C,
			D, E$ & $\phi$ & Prop ($\neg A 
			\lor \neg B$) & 4\\
		$DE\bullet A \neg B$ & $A \lor 
			\neg B, \neg A \lor \neg B, 
			B \lor C, D, E$ & $\phi$ & 
			Prop $(B \lor C)$ & 5\\
		$DE\bullet A \neg B\ C$ & $A \lor 
			\neg B, \neg A \lor \neg B, B 
			\lor C,	D, E$ & $\phi$ & 
			Confl$_{EUF}$ $(\neg C \lor$ 
			& 6\\
		&&&$\neg D \lor \neg E \lor B)$ &\\
		$DE\bullet A \neg B\ C$ & $A \lor 
			\neg B, \neg A \lor \neg B, B 
			\lor C,	D, E$ & $\neg C \lor
			\neg D \lor \neg E \lor	B$ & 
			Expl $(D)$ & 7\\
		$DE\bullet A \neg B\ C$ & $A \lor 
			\neg B, \neg A \lor \neg B, B 
			\lor C,	D, E$ & $\neg C \lor
			\neg E \lor	B$ & Expl $(E)$ & 
			8\\
		$DE\bullet A \neg B\ C$ & $A \lor 
			\neg B, \neg A \lor \neg B, B 
			\lor C, D, E$ & $\neg C \lor B$ 
			& Expl $(B \lor C)$ & 9\\
		$DE\bullet A \neg B\ C$ & $A \lor 
			\neg B, \neg A \lor \neg B, B 
			\lor C,	D, E$ & $B$ & Expl 
			$(\neg A \lor \neg B)$ & 10\\
		$DE\bullet A \neg B\ C$ & $A \lor 
			\neg B, \neg A \lor \neg B, B 
			\lor C, D, E$ & $\neg A$ & 
			Learn $(\neg A)$ & 11\\
		$DE\bullet A \neg B\ C$ & $A \lor 
			\neg B, \neg A \lor \neg B, B 
			\lor C, D, E, \neg A$ & $\neg 
			A$ & Backj $(\neg A)$ & 12\\
		$DE\neg A$ & $A \lor \neg B, \neg 
			A \lor \neg B, B \lor C, D, E, 
			\neg A$ & $\phi$ & Prop $(A 
			\lor \neg B)$ & 13\\
		$DE\neg A \neg B$ & $A \lor \neg B, 
			\neg A \lor \neg B, B \lor C,
			D, E, \neg A$ & $\phi$ & Prop 
			$(B \lor C)$ & 14\\
		$DE\neg A \neg B\ C$ & $A \lor \neg 
			B, \neg A \lor \neg B, B \lor C, 
			D, E, \neg A$ & $\phi$ & 
			Confl$_{EUF}$ $(\neg C \lor$
			& 15\\
		&&&$\neg D \lor \neg E \lor B)$ &\\
		$DE\neg A \neg B\ C$ & $A \lor 
			\neg B, \neg A \lor \neg B, B 
			\lor C,	D, E, \neg A$ & $\neg C 
			\lor \neg D \lor \neg E \lor	
			B$ & Expl $(D)$ & 16\\
		$DE\neg A \neg B\ C$ & $A \lor 
			\neg B, \neg A \lor \neg B, B 
			\lor C,	D, E, \neg A$ & $\neg C \lor
			\neg E \lor	B$ & Expl 
			$(E)$ & 17\\
		$DE\neg A \neg B\ C$ & $A \lor \neg 
			B, \neg A \lor \neg B, B \lor C, 
			D, E, \neg A$ & $\neg C \lor B$ 
			& Expl $(B \lor C)$ & 18\\
		$DE\neg A \neg B\ C$ & $A \lor \neg 
			B, \neg A \lor \neg B, B \lor C, 
			D, E, \neg A$ & $B$ & Expl $(A 
			\lor \neg B)$ & 19\\
		$DE\neg A \neg B\ C$ & $A \lor \neg 
			B, \neg A \lor \neg B, B \lor C, 
			D, E, \neg A$ & $A$ & Expl 
			$(\neg A)$ & 20\\
		$DE\neg A \neg B\ C$ & $A \lor \neg 
			B, \neg A \lor \neg B, B \lor C, 
			D, E, \neg A$ & $\bot$ & Fail & 
			21\\
	\end{tabular}
	\end{center}
	\caption{Execution of the DPLL(T) algorithm on $F$}
	\label{fig:dplltex}
	\end{figure}
	The first and only decision is made in 
	step 3 when $A \to True$ is assigned.
	In step 6, the current assignment is 
	found to be conflicting with theory
	lemma $\neg C \lor \neg D \lor \neg E
	\lor B$, but since this conflict 
	came in a context where a decision
	was made, the decision is reversed
	in Step 12, after the conflict clause 
	is simplified and learned. In step 15, 
	when the same lemma is found to be 
	conflicting with the new context, we 
	can conclude that the input is 
	unsatisfiable using the $Fail$ rule, 
	since we tried both choices of 
	assignment for $A$. This is what 
	the solver would typically do. However, 
	the formula $\bot$ 
	is trivially unsatisfiable and in 
	some given context where 
	$x \to True$ for some $x$, an 
	acceptable proof of unsatisfiability 
	is to to derive from the context that 
	$x \to False$ holds. Such a state 
	where some $x$ and $\neg x$ holds
	logically reduces to $\bot$ 
	($x$ and $\neg x$ resolve to $\bot$).
	Thus, given a set of formulas,
	a more uniform proof of 
	unsatisfiability of the set is 
	to derive $\bot$ from it. This is why, 
	in proof mode, the SMT solver, only 
	applies the final $Fail$ rule after 
	reducing the conflict clause to $\bot$. 
	This is done from step 15 to 21.
	
	\subsection{DPLL(T) Proof Example}
	\label{sec:proofex}
	Now, we extract the proof from the 
	example execution of the transition system.
	Given an execution of a DPLL(T)-based 
	SMT solver that ends in the $fail$ state, 
	we can prove that the input formula 
	$F$ is unsatisfiable by building a 
	\textit{refutation tree} of $F$ --- a 
	tree with the input clauses 
	and theory lemmas at the leaves, 
	that lead, through the application of 
	the resolution rule, to the empty 
	clause $\bot$ at the root. DPLL(T) is 
	a resolution-based calculus since most 
	of its rules perform some form of 
	resolution (and the application of 
	these rules are involved in the 
	refutation tree of the input). For 
	instance, the $Prop$ rule is a 
	resolution of a clause in $F$ and a 
	set of literals in $M$; the $Expl$ 
	rule modifies the current conflict 
	clause by resolving it with an 
	input clause.
	
	Figure~\ref{fig:propproof} shows the 
	proof tree for the unsatisfiability of 
	$F$ from the example in 
	Figure~\ref{fig:dplltex}. The nodes of 
	the tree contain the clause being 
	used. Every reduction of 
	(a pair of) leaves to a child node 
	is an application of the resolution 
	rule. The proof tree is guided by 
	the application of $Confl$ and $Expl$ 
	rules. Every explanation corresponds 
	to a resolution of the explanation
	clause with the current 
	conflict clause. The label 
	of the resolution of two 
	nodes in the tree shows the 
	step number of the application 
	of the $Expl$ from 
	Figure~\ref{fig:dplltex}.
	The left subtree is a proof of 
	$\neg A$,which is learned in steps 
	1 to 11, and the right subtree is the 
	proof of $A$ which is concluded from 
	steps 12 to 20. Step 21 performs 
	the final resolution of $A$ and 
	$\neg A$ to derive $\bot$.
	
	\begin{figure}[t]
		Refutation tree of $F$:
		\begin{prooftree}
			\defaultHypSeparation {\hskip 0in}
			\AxiomC{Proof of $\neg A$}
			\AxiomC{Proof of $A$}
			\RightLabel{20}
			\BinaryInfC{$\bot$}
		\end{prooftree}
		where proof of $\neg A$:
		\begin{prooftree}
			\AxiomC{EUF Proof}
			\UnaryInfC{$\neg C \lor \neg D 
				\lor \neg E \lor B$}
			\AxiomC{$D$}
			\RightLabel{7}
			\BinaryInfC{$\neg C \lor \neg E 
				\lor B$}
			\AxiomC{$E$}
			\RightLabel{8}
			\BinaryInfC{$\neg C \lor B$}
			\AxiomC{$B \lor C$}
			\RightLabel{9}
			\BinaryInfC{$B$}
			\AxiomC{$\neg A \lor \neg B$}
			\RightLabel{10}
			\BinaryInfC{$\neg A$}
		\end{prooftree}
		and proof of $A$:
		\begin{prooftree}
			\AxiomC{EUF Proof}
			\UnaryInfC{$\neg C \lor \neg D 
				\lor \neg E \lor B$}
			\AxiomC{$D$}
			\RightLabel{16}
			\BinaryInfC{$\neg C \lor \neg E 
				\lor B$}
			\AxiomC{$E$}
			\RightLabel{17}
			\BinaryInfC{$\neg C \lor B$}
			\AxiomC{$B \lor C$}
			\RightLabel{18}
			\BinaryInfC{$B$}
			\AxiomC{$A \lor \neg B$}
			\RightLabel{19}
			\BinaryInfC{$A$}
		\end{prooftree}
	\caption{Proof tree for example in 
		Figure~\ref{fig:dplltex}}
	\label{fig:propproof}
	\end{figure}
	Finally, notice that the theory 
	lemma used needs to be justified by
	a proof from the EUF theory solver.
	This is explained in the next 
	section.
	
	\subsection{Theory Lemma Proofs}
	\label{sec:thproof}
	Theory lemmas at the leaves are 
	justified by satellite proofs 
	from the theory solver that produce 
	them, as shown in 
	Figure~\ref{fig:propproof}. The theory 
	solvers that produce the theory lemmas
	are also responsible for justifying 
	these lemmas. So, in the refutation 
	tree of the input produced by the 
	SAT solver, there are theory lemmas 
	at the leaves, which need to be 
	justified by proof trees 
	from the theory solver. These 
	are often proved by contradiction: 
	given a lemma $L$, a proof of $L$, 
	is a derivation of $\bot$ from 
	$\neg L$. The only theory lemma
	used in our example is 
	\begin{align*}
		L :\ & \neg C \lor \neg D \lor 
			\neg E \lor B, \\
		\textrm{or, }L :\ & f(m) \neq f(n) 
			  \lor f(m) \neq x \lor 
			  f(n) \neq y \lor x = y\\
	 \textrm{equivalently, }L :\ &
		f(m) = f(n) \Rightarrow 
		f(m) = x \Rightarrow f(n) = y 
		\Rightarrow x = y
	\end{align*}
	Taking its negation gives:
	\begin{align*}
		\neg L :\ & C \land D \land E \land \neg B\\
			\textrm{or, } \neg L :\ & 
			f(m) = f(n) \land f(m) = x \land
			f(n) = y \land x \neq y
	\end{align*}
	Finaly, the proof tree from the 
	theory solver of $L$, or EUF Proof from 
	Figure~\ref{fig:propproof}, uses the 
	conjuncts of $\neg L$ as leaves --- $C, D, E, 
	\neg B$ --- and derives $\bot$. The proof is 
	shown in Figure~\ref{fig:eufproof}, and it 
	uses the transitivity and symmetry of 
	equality in addition to the resolution
	rule.
	
	\begin{figure}[t]
	\begin{prooftree}
		\AxiomC{$f(m) = x$}
		\RightLabel{$Symmetry$}
		\UnaryInfC{$x = f(m)$}
		\AxiomC{$f(m) = f(n)$}
		\RightLabel{$Transitivity$}
		\BinaryInfC{$x = f(n)$}
		\AxiomC{$f(n) = y$}
		\RightLabel{$Transitivity$}
		\BinaryInfC{$x = y$}
		\AxiomC{$x \neq y$}
		\RightLabel{$Resolution$}
		\BinaryInfC{$\bot$}
	\end{prooftree}
	\caption{EUF Proof from Figure~\ref{fig:propproof}}
	\label{fig:eufproof}
	\end{figure}


	\section{Classical vs Constructive Logic in Coq}
	\label{sec:logic}
	Constructive or intuitionistic logic 
	is a branch of mathematical logic 
	studying arguments concerning 
	constructive objects. A proposition 
	is provable in constructive logic, if 
	a (proof) term inhabiting the 
	proposition can be constructed. 
	Coq implements a constructive 
	logic through its \texttt{Prop}
	type of propositions. In other words,
	a \texttt{Prop} $P$ can be proved in 
	Coq only if a proof term having 
	type $P$ can be constructed.
	For example, the following Coq
	code proves the formula
	$a \Rightarrow a \lor b$, for any 
	\texttt{Prop}s $a$ and $b$.
	\begin{minted}{Coq}
Theorem orIntroL : forall (a b: Prop), a -> a \/ b.
Proof.
  intros a b Ha.
  left. apply Ha.
Qed.
	\end{minted}
	The formula asserts that given a 
	proof of the proposition $a$, we
	have a proof of the 
	disjunction of $a$ and $b$. The 
	\textit{proof script} between 
	\texttt{Proof} 
	and \texttt{Qed} constructs a 
	proof term of type \texttt{forall 
	(a b: Prop), a -> a $\backslash/$ b}. 
	Coq presents to the user a 
	\textit{goal} and a \textit{proof 
	context}, which contains the proof 
	terms available to prove the goal.
	Initially, the goal is the entire 
	proposition being proven, and the 
	context is empty, and the user
	can manipulate the goal to break 
	it down and move over some of its 
	constituents into the context
	using Coq functions called 
	\textit{tactics}. The first line of the 
	script above introduces 
	the propositions $a$ and $b$ and the 
	proof $Ha$ of $a$ into the 
	proof context (using the \texttt{intros}
	tactic). The only way to prove a 
	disjunction is to prove one 
	of the disjuncts. This is a crucial
	detail that distinguishes constructive
	logic from classical logic, as we 
	will see in the next example. 
	The second line of the proof script 
	proves the left disjunct by using 
	the proof of $a$ from the context.
	
	Classical logic extends 
	constructive logic with the 
	\textit{law of the excluded 
	middle} that states that for 
	every proposition $a$, 
	$a \lor \neg a$ holds. This can 
	also be stated as the 
	double negation elimination 
	law or as proving a 
	proposition by contradiction. As
	such, some statements that are true 
	in classical logic might not have 
	a proof in constructive logic.
	The law of the excluded middle
	itself is a convenient example.
	\begin{minted}{coq}
Theorem LEMProp : forall (a : Prop), a \/ ~a.
Proof.
  intros a. 
Abort.
	\end{minted}
	Again, the only way to prove the 
	disjunction is to prove that one 
	of the disjuncts holds. However, 
	we don't any proof terms in our
	context to work 
	with here, to construct the 
	proof of either $a$ or $\neg a$,
	so we quit out of the proof.
	
	SMT solvers are able to prove 
	such classical formulas and as
	such, aren't logically compatible
	with Coq's \texttt{Prop} type. 
	In addition to the \texttt{Prop} 
	type, Coq has a Boolean 
	(\texttt{bool}) type in its standard 
	library~\cite{CoqBool}, that defines 
	Booleans as an 
	inductive type with two nullary
	constructors \texttt{true} and
	\texttt{false}. This Boolean type 
	in Coq implements a classical 
	logic. Any Boolean value can 
	be inspected, pattern matched on,
	and evaluated to either 
	\texttt{true} or \texttt{false}.
	Formulas over the type 
	\texttt{bool} are not theorems
	in Coq, since they don't have 
	the \texttt{Prop} type. To 
	state a Boolean formula in a 
	Coq proposition, the formula is 
	equated to \texttt{true}. 
	For example, the law of 
	excluded middle for 
	\texttt{a : bool} is stated as 
	\begin{center}
		\texttt{a || (negb a)}
	\end{center}
	where \texttt{orb} is the 
	disjunction operator over 
	Booleans and has 
	infix notation \texttt{||},
	and \texttt{negb} is the 
	negation operator over 
	Booleans.
	This is stated as a Coq theorem 
	and proved as follows:
	\begin{minted}{coq}
Theorem LEMBool : forall a : bool, a || (negb a) = true.
Proof.
  intros a. destruct a.
  + simpl. reflexivity.
  + simpl. reflexivity.
Qed.
	\end{minted}
	\texttt{destruct a} performs a 
	case analysis on $a$, and since $a$ 
	is a Boolean that can 
	only be either \texttt{true}
	or \texttt{false}, the two lines
	beginning with \texttt{+} 
	replace the $a$ in the context
	with exactly those values.
	When $a$ is \texttt{true}, the 
	goal reduces to 
	\texttt{true || not true = true}
	and when it is \texttt{false}, it
	reduces to 
	\texttt{false || not false = true}.
	Both of these are simplified to 
	\texttt{true = true} using the 
	tactic \texttt{simpl} and then 
	by the reflexivity of 
	\texttt{true}, both equalities
	are provable. Notice that, 
	neither a proof of $a$, nor 
	a proof of $\neg a$ were 
	constructed here. Arguments were
	made in terms of every possible 
	value $a$ could take, and the
	conjecture was proven in 
	each case. Such a proof, as we 
	saw above, is impossible in 
	constructive logic.
	
	Since SMT solvers implement 
	a classical logic, SMTCoq 
	uses SMT solver proofs to 
	prove classical formulas in 
	Coq over Booleans. The 
	\texttt{LEMBool} lemma from 
	above would be a simple goal that 
	could be dispatched to an SMT 
	solver using SMTCoq's \texttt{smt}
	tactic:
	\begin{minted}{coq}
Theorem LEMBool' : forall a : bool, a || (negb a) = true.
Proof.
  smt.
Qed.
	\end{minted}
	As described in the previous report, 
	SMTCoq defines a \texttt{form} type
	as a \textit{deep embedding} of Boolean 
	formulas in Coq, and \textit{reflects} 
	proofs of \texttt{form}s from the SMT 
	solver to proofs of Boolean formulas, 
	the shallow embedding of formulas 
	in Coq. A checker in Coq checks 
	proofs from the SMT solver against
	the deep embedding of formulas, and 
	the reflection principle is a proof
	of correctness of the checker, that 
	is used by SMTCoq to prove classical
	goals in Coq represented using the
	\texttt{Bool} type.
	
	
	\subsection{Theories and Prop Predicates}
	SMTCoq integrates both 
	SAT and SMT solvers with Coq. If 
	it were reasoning about purely 
	Boolean formulas such as 
	\texttt{LEMBool} above, it wouldn't 
	need SMT solvers at all. It could
	do that with just SAT or
	propositional satisfiability
	solvers. SMT or satisfiability
	modulo theories solvers provide 
	the ability to reason in theories
	such as finite- and infinite-precision 
	integers and arrays. Using its SMT 
	solver integration, SMTCoq is able to 
	prove expressive Boolean formulas
	over the theories of uninterpreted 
	functions, linear integer arithmetic,
	arrays, and bit-vectors. Consider 
	the following examples.
	\begin{minted}{Coq}
Theorem Z_congBool : forall (a b : Z) (f : Z -> Z), 
  (Z.eqb a b) = true -> (Z.eqb (f a) (f b) = true).
Proof.
  smt.
Qed.
	
Theorem BV_ulteZeroBool : forall (b : bitvector 4),
  bv_ult #b|0|0|0|0| b = true \/ bv_eq b #b|0|0|0|0| = true.
Proof.
  smt.
Qed.
	\end{minted}
	\texttt{Z\_congBool} proves the 
	congruence rule over integer 
	to integer functions and 
	\texttt{BV\_ulteZeroBool} proves 
	that all 4-bit integers (bit-vectors 
	of length 4) in their 
	unsigned interpretation are 
	greater than or equal to $0$. 
	\texttt{Z.eqb} is the Boolean 
	equality predicate over integers 
	in Coq, and \texttt{bv\_ult}
	and \texttt{bv\_eq} are 
	respectively the Boolean 
	less than (over unsigned integers)
	and equality predicate for 
	machine integers. Since 
	predicates in these theories 
	are expressed as Boolean terms, 
	they need to be equated to 
	\texttt{true} as shown above to 
	be well-formed propositions in 
	Coq. However, for some of these 
	predicates, SMTCoq is able to 
	prove their \texttt{Prop} version,
	in addition to the \texttt{bool}
	version. In other words, SMTCoq
	is able to prove certain 
	non-Boolean propositions in Coq.
	To do this, it implements a 
	\texttt{Prop} to \texttt{bool}
	conversion for these predicates, 
	so when a user asks SMTCoq to 
	prove a proposition over one 
	of these predicates, SMTCoq
	converts it to a Boolean 
	predicate, so that it now
	suffices to prove the formula
	over the Boolean predicate, 
	which can be done by an SMT 
	solver. As it turns out, 
	the \texttt{Prop} to 
	\texttt{Bool} conversion has 
	been implemented for the 
	predicates used above,
	so SMTCoq can prove their 
	corresponding \texttt{Prop}
	theorems:
	\begin{minted}{Coq}
Theorem Z_congProp : forall (a b : Z) (f : Z -> Z), a = b -> 
  f a = f b.
Proof.
  smt.
Qed.
	
Theorem BV_ulteZeroProp : forall (b : bitvector 4), 
  bv_ultP #b|0|0|0|0| b \/ b = #b|0|0|0|0|.
Proof.
  smt.
Qed.
	\end{minted}
	The	\texttt{Prop} to \texttt{Bool} 
	conversion of SMTCoq consists of 
	applying proofs of equivalence 
	between a \texttt{Prop} version
	of a predicate and its \texttt{Bool}
	version. For example, SMTCoq's 
	bitvector library proves the 
	following lemma of equivalence 
	between the \texttt{Bool}
	and \texttt{Prop} version of the 
	unsigned less than predicate over
	bit-vectors, which is applied 
	by SMTCoq's \texttt{Prop} to 
	\texttt{Bool} conversion when 
	a user uses the \texttt{Prop}
	version of this predicate in a 
	query to SMTCoq, as in 
	\texttt{BV\_ulteZeroProp}.
	\begin{minted}{Coq}
Theorem bv_ult_B2P : forall n (a b : bitvector n), 
  bv_ult a b = true <-> bv_ultP a b.
	\end{minted}
	The \texttt{Prop} to \texttt{Bool}
	conversion is possible for a predicate 
	on a type, if that predicate is 
	decidable --- that is, for every 
	element of that type, the predicate 
	either holds or doesn't hold. This 
	decidability is proved via a lemma 
	such as \texttt{bv\_ult\_B2P}. Some
	of the lemmas used by the 
	\texttt{Prop} to \texttt{Bool} 
	conversion module are proved in SMTCoq's
	libraries, others are proved in 
	Coq's standard libraries. For 
	example, Coq's library of integers
	proves the equivalence between the 
	Boolean and Propositional version 
	of the less than predicate.
	\begin{minted}{Coq}
Lemma ltb_lt (n m : Z) : (n <? m) = true <-> n < m.
	\end{minted}
	The module in SMTCoq that does
	the conversion applies these 
	equivalence lemmas to reduce a 
	\texttt{Prop} predicate to a 
	\texttt{Bool} one, before sending 
	it to the SMT solver.	
	
	
\bibliographystyle{abbrv}
\bibliography{bib3}

\end{document}